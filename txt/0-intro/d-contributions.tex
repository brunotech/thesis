%!TEX root = ../../THESIS.tex

\section*{Contributions}
\addcontentsline{toc}{section}{Contributions}

The first chapter is an extended version of the DisCoPy paper~\cite{FeliceEtAl20a}.
It emerged from a dialectic teacher-student collaboration with Giovani de Felice: implementing our own category theory library was a way to teach him Python programming.
Bob Coecke then added the capital letters to the name of DisCoPy.
\begin
{itemize}
\item We\footnote
{The ``we'' of this section refers to the author of this thesis.
Although we believe that science is collaboration and that the notion of personal contribution is obsolete, it is in fact required by university regulations: ``Where some part of the thesis is not solely the work of the candidate or has been carried out in collaboration with one or more persons, the candidate shall submit a clear statement of the extent of his or her own contribution.''} give an elementary definition of string diagrams for monoidal categories.
Our construction decomposes the free monoidal category construction into three basic steps: 1) a layer monad on the category of monoidal signatures, 2) the free premonoidal category as a free category of layers and 3) the free monoidal category as a quotient by interchangers.
To the best of our knowledge, this \emph{premonoidal approach} had been relegated to mathematical folklore: it was known by those who knew it, yet it never appeared in print.
\item We prove the equivalence between our elementary definition and the topological definition of Joyal and Street~\cite{JoyalStreet88}.
One side of this equivalence underlies the drawing algorithm of DisCoPy, the other side is the basis of a prototype for an automatic diagram recognition algorithm.
\item We describe our object-oriented implementation of monoidal category theory.
The hierarchy of categorical structures (monoidal, closed, rigid, etc.) is encoded in a hierarchy of Python classes and an inheritance mechanism implements the free-forgetful adjunctions between them.
\item We discuss the relationship between our premonoidal approach and the existing graph-based data structures for diagrams in symmetric monoidal categories.
\end
{itemize}
The second chapter deals with QNLP, building on \cite{MeichanetzidisEtAl20,CoeckeEtAl20,MeichanetzidisEtAl20a}.
It was joint work with Bob Coecke, Giovanni de Felice and Konstantinos Meichanetzidis.
Although we were working in the same office, Stefano Gogioso arrived at the same ideas independently with his collaborator Nicolò Chiappori.
\begin
{itemize}
\item We define QNLP models as functors from grammar to quantum circuits and show that any DisCoCat model can be implemented in this way.
\item We develop a rewriting strategy for the resulting circuits which reduces both the required number of qubits and the amount of post-selection.
The underlying algorithm, called \emph{snake removal}, computes the normal form of diagrams in rigid monoidal categories.
\item We introduce a hybrid classical-quantum algorithm to train QNLP models on a question-answering task.
The underlying idea of \emph{functorial learning}, i.e. learning structure-preserving functors from diagram-like data, provides a theoretical framework for machine learning on structured data.
\end
{itemize}
The third chapter introduces \emph{diagrammatic differentiation}, a graphical calculus for computing the gradients of parameterised diagrams which applies to the training of QNLP models but also to functorial learning in general.
Most of the material has been published in joint work with Richie Yeung and Giovanni de Felice~\cite{ToumiEtAl21a}.
\begin
{itemize}
\item We generalise the dual number construction from rings to monoidal categories. Dual diagrams are formal sums of a string diagram (the real part) and its derivative with respect to some parameter (the epsilon part).
\item We introduce graphical gadgets called bubbles, which can encode arbitrary unary operators on monoidal categories.
In particular, they encode differentiation of diagrams and allow to express the standard rules of calculus (linearity, product, chain) entirely in terms of diagrams.
\item We study diagrammatic differentiation for the ZX calculus.
In the pure case, this allows to compute the gradients of linear maps with respect to phase parameters.
In the mixed classical-quantum case, this yields a definition of the parameter-shift rules used in quantum machine learning.
\item We define the gradient of QNLP models and parameterised functors in general.
\end{itemize}

\section*{Publications}
\addcontentsline{toc}{section}{Publications}

The material presented in this thesis builds on the following publications.
\begin{itemize}[label={}]
\item \enumcite{MeichanetzidisEtAl20a}\vspace{-10pt}
\item \enumcite{FeliceEtAl20a}\vspace{-10pt}
\item \enumcite{CoeckeEtAl20}\vspace{-10pt}
\item \enumcite{MeichanetzidisEtAl20}\vspace{-10pt}
\item \enumcite{ToumiEtAl21a}
\end{itemize}\vspace{10pt}
During his DPhil, the author has also published the following articles.
\begin{itemize}[label={}]
\item \enumcite{FeliceEtAl19}\vspace{-10pt}
\item \enumcite{FeliceEtAl20}\vspace{-10pt}
\item \enumcite{ShieblerEtAl20}
\item \enumcite{KartsaklisEtAl21}\vspace{-10pt}
\item \enumcite{ToumiKoziell-Pipe21}
\item \enumcite{CoeckeEtAl21}\vspace{-10pt}
\item \enumcite{McPheatEtAl21}
\end{itemize}
