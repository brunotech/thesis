%!TEX root = ../../THESIS.tex

\section*{How can category theory help?}
\addcontentsline{toc}{section}{How can category theory help?}

\justepigraph{
I should still hope to create a kind of \emph{universal symbolistic} (\emph{spécieuse générale}) in which all truths of reason would be reduced to a kind of calculus.
}{\emph{Letter to Nicolas Remond}, Leibniz (1714)}

``Every sufficiently good analogy is yearning to become a functor''~\cite{Baez06} and we will see that the analogy behind DisCoCat models is indeed a functor.
Coecke et al.~\cite{CoeckeEtAl13} make a meta-analogy between their models of natural language and \emph{topological quantum field theories} (TQFTs).
Intuitively, there is an analogy between regions of spacetime and quantum processes: both can be composed either in sequence or in parallel.
TQFTs formalise this analogy: they assign a quantum system to each region of space and a quantum process to each region of spacetime, in a way that respects sequential and parallel composition.
In the same structure-preserving way, DisCoCat models assign a vector space to each grammatical type and a linear map to each grammatical derivation.
Both TQFTs and DisCoCat can be given a one-sentence definition in terms of category theory: they are examples of \emph{functors into the category of vector spaces}.

How can the same piece of general abstract nonsense (category theory's nickname) apply to both quantum gravity and natural language processing?
And how can this nonsense be of any help in the implementation of QNLP algorithms?
This section will answer with a history of category theory and its applications to quantum physics and computational linguistics, from an abstract framework for meta-mathematics to a concrete toolbox for NLP on quantum hardware.
First, a short philosophical digression on the etymology of the words ``functor'' and ``category'' shall bring some light to their divergent meanings in mathematics and linguistics.

The word ``functor'' first appears in Carnap's \emph{Logical syntax of language}~\cite{Carnap37} to describe what would be called a \emph{function symbol} in a modern textbook on first-order logic.
He introduces them as a way to reduce the laws of empirical sciences like physics to the pure syntax of his formal logic, taking the example of a \emph{temperature functor} $T$ such that $T(3) = 5$ means ``the temperature at position 3 is 5''\footnote
{MacLane~\cite{MacLane38} would later remark that Carnap's formal language cannot express the coordinate system for positions, nor the scale in which temperature is measured.}.
This meaning has then drifted to become synonymous with \emph{function words} such as ``such'', ``as'', ``with'', etc. These words do not refer to anything in the world but serve as the grammatical glue between the \emph{lexical words} that describe things and actions.
They represent less than one thousandth of our vocabulary but nearly half of the words we speak~\cite{ChungPennebaker07}.

Categories (from the ancient Greek \emph{\foreignlanguage{greek}{κατηγορία}}, ``that which can be said'') have a much older philosophical tradition.
In his \emph{Categories}, Aristotle first makes the distinction between the simple forms of speech (the things that are ``said without any combination'' such as ``man'' or ``arguing'') and the composite ones such as ``a man argued''.
He then classifies the simple, atomic things into ten categories: ``each signifies either substance or quantity or qualification or a relative or where or when or being-in-a-position or having or doing or being-affected''.
A common explanation~\cite{Ryle37} for how Aristotle arrived at such a list is that it comes from the possible \emph{types of questions}: the answer to ``What is it?'' has to be a substance, the answer to ``How much?'' a quantity, etc.
Although he was using language as a tool, his system of categories aims at classifying things in the world, not forms of speech: it was meant as an \emph{ontology}, not a grammar.
In his \emph{Critique of Pure Reason}~\cite{Kant81}, Kant revisits Aristotle's system to classify not the world, but the mind: he defines categories of understanding rather than categories of being.
The idea that every object (whether in the world or in the mind) is an object of a certain type has then become foundational in mathematical logic and Russell's \emph{theory of types}~\cite{Russell03}.
The same idea has also had a great influence in linguistics and especially in the \emph{categorial grammar} tradition initiated by Ajdukiewicz~\cite{Ajdukiewicz35} and Bar-Hillel~\cite{Bar-Hillel53,Bar-Hillel54}, where categories have now become synonymous with \emph{grammatical types}.
As we shall see in section~\ref{subsection:lambek-discocat}, the key innovation from Aristotelian categories to categorial grammars is that the grammatical types now come with some structure: we can compose \emph{atomic categories} together to form complex types, confusingly called \emph{functor categories}.

Independently of their use in linguistics, a series of papers from Eilenberg and MacLane~\cite{EilenbergMacLane42,EilenbergMacLane42a,EilenbergMacLane45} gave categories and functors their current mathematical definition.
Inspired by Aristotle's categories of things and Kant's categories of thoughts, they defined categories as types of \emph{mathematical structures}: sets, groups, spaces, etc.
Their great insight was to focus not on the content of the objects (elements, points, etc.) but on the composition of the \emph{arrows} between them: functions, homomorphisms, continuous maps, etc.
Applying the same insight to categories themselves, what really matters are the arrows between them: \emph{functors}, maps from one category to another that preserve the form of arrows.\footnote
{We can play the same game again: what matters are not so much the functors themselves but the \emph{natural transformations} between them, which is what category theory was originally meant to define.
To keep playing that game is to fall in the rabbit hole of infinity category theory~\cite{RiehlVerity16}.}
A prototypical example is Poincaré's construction of the fundamental group of a topological space~\cite{Poincare95}, which can be defined as a functor from the category of (pointed) topological spaces to that of groups: every continuous map between spaces induces a homomorphism between their fundamental groups, in a way that respects composition and identity.
Thus, the abstraction of category theory allowed to formalise the analogies between topology and algebra, proving results about one using methods from the other.
It was then used as a tool for the foundation of algebraic geometry by the school of Grothendieck~\cite{GrothendieckDieudonne60}, which brought the analogy between geometric shapes and algebraic equations to a new level of abstraction and led to the development of \emph{topos theory}.

The establishment of category theory as an independent discipline and as a foundation for mathematics owes much to the work of Lawvere.
His influential Ph.D. thesis~\cite{Lawvere63} on \emph{functorial semantics} set up a framework for model theory where logical theories are categories and their models are functors.
He then undertook the axiomatisation of the category of sets~\cite{Lawvere64} and the category of categories~\cite{Lawvere66}.
The resulting notion of elementary topos~\cite{Lawvere70a} subsumed Grothendieck's definition and emphasised the foundational concept of \emph{adjunction}~\cite{Lawvere69a,Lawvere70}.
``Adjoint functors arise everywhere'' became the slogan of MacLane's classic textbook \emph{Categories for the working mathematician}~\cite{MacLane71}.
Lambek~\cite{Lambek68,Lambek69,Lambek72} used the related notion of \emph{cartesian closed categories} to extend the Curry-Howard correspondance between logic and computation into a trinity with category theory: proofs and programs are arrows, logical formulae and data types are objects.
The discovery of this three-fold connection resulted in a wide range of applications of category theory to theoretical computer science, surveyed in Scott~\cite{Scott00}.

This unification of mathematics, logic and computer science has been followed by a program for the categorical foundations for physics, initiated by Lawvere's topos-theoretic treatment of classical dynamics~\cite{Lawvere79} and continuum physics~\cite{LawvereSchanuel86} with Schanuel.
As we mentioned at the start of this section, the work of Atiyah~\cite{Atiyah88}, Baez and Dolan~\cite{BaezDolan95} on TQFTs showed categories and functors to be essential tools in the grand unification project of quantum gravity~\cite{Baez06}.
This now quaternary analogy between physics, mathematics, logic and computation was popularised by Baez and Stay in their \emph{Rosetta Stone}~\cite{BaezStay10}.
On more concrete grounds, this connection between category theory and quantum physics appeared in Selinger's proposal of a quantum programming language~\cite{Selinger04} and the development of a quantum lambda calculus~\cite{VanTonder04,SelingerValiron06,SelingerEtAl09}.
The same insight blossomed in the school of \emph{categorical quantum mechanics} (CQM) led by Abramsky and Coecke~\cite{AbramskyCoecke04}, where quantum processes are arrows in \emph{compact closed categories}.
This approach culminated in the \emph{ZX calculus} of Coecke and Duncan~\cite{CoeckeDuncan08,CoeckeDuncan11}, a categorical axiomatisation which was proved complete for qubit quantum computing~\cite{JeandelEtAl18a,HadzihasanovicEtAl18}
with applications including error correction \cite{ChancellorEtAl18,GidneyFowler19}, circuit optimisation~\cite{KissingerVanDeWetering20,DuncanEtAl20,DeBeaudrapEtAl20}, compilation \cite{CowtanEtAl20,DeGriendDuncan20} and extraction \cite{BackensEtAl21}.

In quantum computing as well, adjunction is fundamental: it underlies the definition of entanglement and the proof of correctness for the \emph{teleportation protocol}.
Back in 2004 when Coecke first presented this result at the McGill category theory seminar, Lambek immediately pointed out the analogy with his \emph{pregroup grammars}~\cite{Lambek99,Lambek01} where adjunction is the only grammatical rule\footnote
{See \cite{Coecke21} for a first-hand account of this story and a praise of Jim Lambek.}.
Half a century beforehand, the \emph{Lambek calculus}~\cite{Lambek58,Lambek59,Lambek61} revealed an analogy between the derivations in categorial grammars and proof trees in mathematical logic.
He then extended this analogy in \emph{Categorial and categorical grammar}~\cite{Lambek88} where he showed that these grammatical derivations are in fact arrows in \emph{closed monoidal categories} and proposed to cast Montague semantics as a topos-valued functor.
Later, he argued not ``that categories should play a role in linguistics, but rather that they already do''~\cite{Lambek99b}.
Indeed, Hotz~\cite{Hotz66} had already proved that Chomsky's generative grammars were \emph{free monoidal categories}, although his original German article was never translated to English.
The idea of using functors as semantics had appeared implicitly in Knuth~\cite{Knuth68a} in the context-free case and was made explicit by Benson~\cite{Benson70a} for unrestricted grammars.
From this categorical formulation of linguistics, Lambek~\cite{Lambek10} first suggested the analogy between linguistics and physics which is the basis of this thesis: \emph{pregroup reductions as quantum processes}.

It is remarkable that Lambek could foresee QNLP without \emph{string diagrams}\footnote
{String diagrams do not appear in any of Lambek's published work.
Instead, he either uses lines of equations, proof trees or ``underlinks'' for pregroup adjunctions~\cite{Lambek08}.
He admits ``not having had the patience to absorb'' the topological definition of Joyal-Street string diagrams~\cite{Lambek10}.
}, probably the most powerful tool in the hands of the applied category theorist.
They first appeared in another article from Hotz~\cite{Hotz65} as a formalisation of the diagrams commonly used in electronics.
Penrose~\cite{Penrose71} then used the same notation as an informal shortcut for tedious tensor calculations, and later applied it to relativity theory with Rindler~\cite{PenroseRindler84}.
Joyal and Street~\cite{JoyalStreet88,JoyalStreet91,JoyalStreet95} gave the first topological definition of string diagrams and characterised them as the arrows of free monoidal categories.
A generalisation of string diagrams called \emph{proof nets} were introduced by Girard~\cite{Girard87} as a way to free the proofs of his \emph{linear logic} from ``the bureaucracy of syntax'', they were then applied to the Lambek calculus~\cite{Roorda92} and to its multimodal extensions~\cite{MootPuite02}.

At first a piece of mathematical folklore that was hand-drawn on blackboards and rarely included in publications, string diagrams were published at a much bigger scale with the advent of typesetting tools like \LaTeX \ and Ti\emph{k}Z.
Selinger's survey~\cite{Selinger10}, makes the hierarchy of categorical structures (symmetric, compact closed, etc.) correspond to a hierarchy of graphical gadgets (swaps, wire bending, etc.).
In \emph{Picturing Quantum Processes}~\cite{CoeckeKissinger17}, Coecke and Kissinger introduce quantum theory with over a thousand diagrams.
And the list of applications keeps growing:
electronics~\cite{BaezFong15} and chemistry~\cite{BaezPollard17},
control theory~\cite{BaezErbele14} and concurrency~\cite{BonchiEtAl14a},
databases~\cite{BonchiEtAl18} and knowledge representation \cite{Patterson17},
Bayesian inference~\cite{CoeckeSpekkens12,ChoJacobs19} and causality~\cite{KissingerUijlen19},
cognition~\cite{BoltEtAl17} and game theory~\cite{GhaniEtAl18},
functional programming \cite{Riley18} and machine learning~\cite{FongEtAl17}.

If they are a great tool for writing scientific papers, string diagrams can also be a powerful data structure for developing software applications:
quantomatic~\cite{KissingerZamdzhiev15} and its successor PyZX~\cite{KissingerVanDeWetering19} perform automatic rewriting of diagrams in the ZX calculus,
globular~\cite{BarEtAl18} and its successor homotopy.io~\cite{ReutterVicary19} are proof assistants for higher category theory,
cartographer~\cite{SobocinskiEtAl19} and catlab~\cite{PattersonEtAl21} implement diagrams in symmetric monoidal categories, which are also implicit in the circuit data structure of the t$|$ket$\rangle$ compiler~\cite{SivarajahEtAl20}.
String diagrams are the main data structure of our QNLP algorithms: we translate the diagrams of sentences into diagrams of quantum circuits.
As none of the existing category theory software was flexible enough, we had to implement our own: DisCoPy~\cite{FeliceEtAl20}, a Python library for computing with functors and diagrams in monoidal categories.
DisCoPy then became the engine underlying lambeq~\cite{KartsaklisEtAl21}, a high-level library for experimental QNLP.
Although its development was driven by the implementation of DisCoCat models on quantum computers, DisCoPy was designed as a general-purpose toolkit for applied category theory.
It is freely available\footnote{\url{https://github.com/oxford-quantum-group/discopy}} (as in free beer and in free speech), reliable (with 100\% code coverage) and extensively documented\footnote{\url{https://discopy.readthedocs.io/}}.

In conclusion, category theory can really be a \emph{theory of anything}: from algebraic geometry and quantum gravity to natural language processing.
There is a striking analogy between category theory and string diagrams as a universal graphical language and the \emph{characteristica universalis} and \emph{calculus ratiocinator} dreamt by Leibniz three hundred years ago, a formal language and computational framework that would be able to express all of mathematics, science and philosophy.
Indeed, not only can categories be tools for the working mathematicians and scientists, they can also be of help to the philosophers.
In the footsteps of Grassmann's \emph{Ausdehnungslehre}~\cite{Grassmann44} and his project of an algebraic formalisation of Hegel, Lawvere~\cite{Lawvere89,Lawvere91,Lawvere92,Lawvere96} set out to formulate Hegelian dialectics in terms of adjunctions.
This led to the ongoing effort of Schreiber, Corfield and their collaborators on the nLab~\cite{SchreiberEtAl21} to translate \emph{Wissenschaft Der Logik}~\cite{Hegel12} in terms of category theory.
Not only can it accommodate the absolute idealism of Hegel, category theory can also deal with the pragmatism of Peirce~\cite{Peirce06},
who developed first-order logic independently of Frege using what was later recognised as the first string diagrams~\cite{BradyTrimble98,BradyTrimble00,MelliesZeilberger16,HaydonSobocinski20}.
String diagrams have also been used to model Wittgenstein's language games as functors from a grammar to a category of games~\cite{HedgesLewis18}.
In recent work~\cite{FeliceEtAl20a}, we applied these functorial language games to question answering, going from philosophy to NLP via category theory.
