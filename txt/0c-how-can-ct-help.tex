%!TEX root = ../THESIS.tex

\section*{How can category theory help?}
\addcontentsline{toc}{section}{How can category theory help?}

\justepigraph{
A striking aspect of the notation is that it is pictorial rather than sequential or alphabetical. This made it difficult to print, which partly explains why no rigorous theory was developed.
}{\textit{The Geometry of Tensor Calculus},\\
Joyal and Street (1991)}

``Every sufficiently good analogy is yearning to become a functor''~\cite{Baez06} and indeed, DisCoCat models are functors.
Coecke et al.~\cite{CoeckeEtAl13} make a meta-analogy between DisCoCat models and \emph{topological quantum field theories} (TQFTs).
Intuitively, there is an analogy between regions of spacetime and quantum processes: both can be composed either in sequence or in parallel.
TQFTs formalise this analogy: they assign a quantum system to each region of space and a quantum process to each region of spacetime, in a way that respects sequential and parallel composition.
DisCoCat models assign a vector space to each grammatical type and a linear map to each grammatical derivation in the same structure-preserving way.
Both TQFTs and DisCoCat can be given a one-sentence definition in the language of category theory: they are examples of functors into the category of vector spaces.

How can the same general abstract nonsense\footnote
{Before category theory was introduced, it was half-jokingly called ``general abstract nonsense''~\cite{Lane97}.}
apply both to quantum gravity and to natural language processing?
And what is a functor exactly?


Category theory as the mathematics of analogy, started by studying the analogies of mathematics.

\begin{itemize}
    \item Arguably the first instance of a functor, group extensions \cite{Baer34}
    \item The word ``functor'' \cite{Carnap37} was only introduced three years later.
    \item $Hom$ and $Ext$ \cite{EilenbergMacLane42a}
    \item natural transformation \cite{EilenbergMacLane42b}
    \item categories and functors \cite{EilenbergMacLane45}

\end{itemize}

Category theory starts in 1945 with Eilenberg and MacLane~\cite{EilenbergMacLane45} formalising what had become a colloquial expression among algebraic topologists: \emph{natural transformations}, which are usually depicted as square \emph{commutative diagrams}:
$$FX \to FY$$
$$GX \to GY$$
In order to define them, it was necessary to define \emph{functors}, which is what a natural transformation is transforming, borrowing the word from Carnap.
In order to define functors, it was necessary to define \emph{categories}, which is what functors themselves are transforming, borrowing the word from Aristotle.
