%!TEX root = ../../THESIS.tex

\subsection{Categorial grammars and Montague semantics}

Even if we cannot get rid of the inherent ambiguity of natural language, we can still try to reduce the artificial ambiguity of our grammar formalism, i.e. the number of weakly equivalent grammars that generate the same language.

The \emph{categorial grammar} tradition may be summed up in a slogan: \emph{all the grammar is in the dictionary}~\cite{Preller07}.
Indeed, there is no need for language-specific production rules if the types of our grammar have enough structure, if we go from monoidal to closed categories.
In the \emph{Lambek calculus}~\cite{Lambek58}, a categorial grammar is defined as a tuple $G = (V, X, R, s)$ where:
\begin{itemize}
\item $V$ and $X$ are finite sets called the \emph{vocabulary} and the \emph{basic types} with $s \in X$ the \emph{sentence type},
\item $R \sub V \times T(X)$ is a finite set of \emph{dictionary entries} with $T(X) \supseteq X$ the set of formal expressions with $1, (x \otimes y), (x / y), (x \backslash y) \in T(X)$ for all $x, y \in T(X)$.
\end{itemize}
Equivalently, the Lambek grammar $G$ may be seen as a closed monoidal signature (as defined in section~\ref{subsection:closed}) with dictionary entries as boxes where the domain is a single word\footnote
{Note that the original calculus did not include a unit for the tensor product, here we follow the presentation given by Lambek~\cite{Lambek88} thirty years later.}.
In a \emph{basic categorial grammar}, also called an AB grammar after Ajdukiewicz~\cite{Ajdukiewicz35} and Bar-Hillel~\cite{Bar-Hillel54}, the dictionary is restricted to a closed signature, i.e. types are generated without the tensor product and unit.
The language of a categorial grammar $G$ is given by $L(G) = \{ w \in V^\star \ \vert \ \exists \ f : w \to s \in \G \}$ for $\G$ the free closed category generated by the dictionary.
More explicitly, a grammatical structure $f : w_1 \dots w_n \to s$ is given by a tensor of dictionary entries $(w_i, t_i) \in R$ followed by a closed diagram $t_1 \dots t_n \to s$ composed only of evaluation and currying.
Traditionally, these closed diagrams have been defined in terms of a \emph{sequent calculus} à la Gentzen, see Lambek~\cite{Lambek88} for a translation between the two definitions.

\begin{python}
{\normalfont Implementation of categorial grammars as closed categories.}

\begin{minted}{python}
class Parsing(closed.Diagram, cfg.Parsing): pass
class Ev(closed.Ev, Parsing): pass
class Word(cfg.Word, Parsing): pass
Ev.upgrade = Word.upgrade = Parsing.upgrade
\end{minted}
\end{python}

\begin{example}
We can take $X = \{ s, n, np \}$ and assign common noun the type $n$, determinants $(np \backslash n)$ and transitive verbs $((np / s) \backslash np)$.

\begin{minted}{python}
n, np, s = map(Ty, ('n', 'np', 's'))
man, island = (Word(noun, n) for noun in ("man", "island"))
no, an = (Word(determinant, np << n) for determinant in ("no", "an"))
_is = Word("is", (np >> s) << np)

no_man_is_an_island = no @ man @ _is @ an @ island\
    >> Ev(np << n) @ ((np >> s) << np) @ Ev(np << n)\
    >> np @ Ev((np >> s) << np) >> Ev(np >> s)

no_man_is_an_island.draw()
\end{minted}

\ctikzfig{img/nlp/no-man-is-an-island}
\end{example}

Bar-Hillel et al.~\cite{Bar-HillelEtAl60} showed that basic categorial grammars are strongly equivalent to context-free grammars in \emph{Greibach normal form}~\cite{Greibach65}, where every production has the form $x \to w y$ for a non-terminal $x \in X$, a word $w \in V$ and a string of non-terminals $y \in X^\star$.
Thus, their parsing problem can be solved in polynomial time.
Pentus~\cite{Pentus93} then showed that Lambek grammars are weakly equivalent to CFGs as well, although their parsing problem is $\mathtt{NP}$-complete~\cite{Pentus06}.
This means that unless $\mathtt{P} = \mathtt{NP}$ there are Lambek grammars for which the smallest weakly equivalent CFG will have exponential size.
Many extensions of the Lambek calculus have been introduced to go beyond its context-free limitation and give a more fine-grained description of syntactic phenomena, see Moortgat~\cite{Moortgat14} for a survey.
Additional unary operators called \emph{modalities}\footnote
{There is no consensus on what the definition of modality should be: ``Ask three modal logicians what modal logic is, and you are likely to get at least three different answers''~\cite{BlackburnEtAl02}
In most cases however, modalities are \emph{comonads}~\cite{CirsteaEtAl11}.} allow to break away from the planarity and linearity of closed diagrams, introducing rules for swaps and comonoids in a controlled way to model phenomena such as \emph{parasitic gaps}, \emph{ellipsis} and \emph{anaphora}.
See McPheat et al.~\cite{McPheatEtAl21} where the present author and collaborators introduce a diagrammatic syntax and functorial semantics for such modalities.
The \emph{combinatory categorial grammars} (CCGs) of Steedman~\cite{Steedman87,Steedman00} take a different approach inspired by the combinatory logic of Schönfinkel~\cite{Schonfinkel24} and Curry~\cite{Curry30}, a variable-free predecessor to the lambda-calculus.
In particular, CCGs include \emph{crossed composition} rules which make them mildly context-sensitive, see Kartsaklis and Yeung~\cite{YeungKartsaklis21} for their implementation in DisCoPy.

% Montague semantics as functor into free cartesian closed category with predicates as boxes
