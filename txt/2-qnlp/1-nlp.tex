%!TEX root = ../../THESIS.tex

\section{A very short introduction to QNLP}\label{section:NLP}

The previous chapter has put much emphasis on string diagrams and its role at the intersection of mathematics and computer science.
From the programming perspective, diagrams are a two-dimensional generalisation of lists which may describe the run of a Turing machine, the syntax of a first-order logic formula or the architecture of a neural network.
In fact, we will see that string diagrams also play a key role in linguistics, where they allow to encode the grammatical structure of sentences.
First, section~\ref{subsection:chomsky} reviews Chomskyan grammars, the notion of ambiguity and the computational complexity of parsing.
Then we discuss categorial grammars, from the Lambek calculus and Montague semantics to pregroup grammars and DisCoCat models.
Finally, we summarise previous work on the Frobenius anatomy of anaphora and investigate the quantum complexity of DisCoCat models.

%
% \subsection{Dependency grammars}
%
% \cite{Gaifman65}
%
% as context-free grammars
%
% as pregroup grammars with \emph{functional} or \emph{operadic} signatures
