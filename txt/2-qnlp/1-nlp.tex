%!TEX root = ../../THESIS.tex

\section{Natural language processing with DisCoPy}\label{section:NLP}

The previous chapter has put much emphasis on string diagrams and its role at the intersection of mathematics and computer science.
From the programming perspective, diagrams are a two-dimensional generalisation of lists which may describe the run of a Turing machine, the syntax of a first-order logic formula or the architecture of a neural network.
In fact, we will see that string diagrams also play a key role in linguistics, where they allow to encode \emph{grammatical structure}.

%
% \subsection{Dependency grammars}
%
% \cite{Gaifman65}
%
% as context-free grammars
%
% as pregroup grammars with \emph{functional} or \emph{operadic} signatures
