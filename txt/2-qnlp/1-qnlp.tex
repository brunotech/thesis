%!TEX root = ../../THESIS.tex

\chapter{Quantum natural language processing} \label{chapter-2:qnlp}

This chapter introduces quantum natural language processing (QNLP) models as monoidal functors from grammar to quantum circuits.
Building on the previous chapter, we show how to implement QNLP models in DisCoPy and how to train them to solve NLP tasks such as classification and question answering.

\section{Formal grammars and quantum complexity}\label{section:NLP}

The previous chapter has put much emphasis on string diagrams and its role at the intersection of mathematics and computer science.
From the programming perspective, diagrams are a two-dimensional generalisation of lists which may describe the run of a Turing machine, the syntax of a first-order logic formula or the architecture of a neural network.
In fact, we will see that string diagrams also play a key role in linguistics, where they allow to encode the grammatical structure of sentences.
First, section~\ref{subsection:chomsky} reviews formal grammars, the notion of ambiguity and the computational complexity of parsing.
Then we discuss categorial grammars, from the Lambek calculus and Montague semantics to pregroup grammars and DisCoCat models.
Finally, we summarise previous work on the Frobenius anatomy of anaphora and investigate the quantum complexity of DisCoCat models.
