%!TEX root = ../../THESIS.tex

\subsection{Quantum channels and mixed quantum circuits}\label{section:mixed-circuits}

DisCoPy implements a variant of the definition of \emph{classical-quantum maps} (cq-maps) from Coecke and Kissinger~\cite[Chapter~8]{CoeckeKissinger17}.
Abstractly, it is a simplification of the $\mathbf{CP^\star}$ construction from Coecke, Heunen and Kissinger~\cite{CoeckeEtAl14a}, which itself generalises the notion of finite-dimensional C$^\star$-algebra to arbitrary dagger compact closed categories.
Concretly, we first define the category $\mathbf{CQMap}$ with objects given by pairs of natural numbers\footnote
{We actually implement an equivalent category where objects are pairs of lists of natural numbers and the arrows are tensors rather than matrices.} $(a, b) \in \N \times \N$ for the classical and quantum dimensions of the system.
The arrows $f : (a, b) \to (c, d)$ are given by $(a b^2) \times (c d^2)$ complex matrices, with composition given by matrix multiplication and tensor given by the following diagram:
\ctikzfig{img/qnlp/channel-tensor}
A \emph{quantum channel} is a cq-map subject to the following two conditions:
\begin{itemize}
\item \emph{complete positivity} (CP), there is a dimension $n \in \N$ and an $(a b) \times (c d n)$ matrix $g$ with element-wise conjugate $g^\star$ such that:
\ctikzfig{img/qnlp/complete-positivity}
\item \emph{trace preservation} (TP) also called \emph{causality}:
\ctikzfig{img/qnlp/trace-preservation}
\end{itemize}
Note that we take the convention to use the \emph{algebraic conjugate} which is the identity on objects, rather than the \emph{diagrammatic conjugate} which reverses the order of wires.
This makes the implementation easier at the cost of breaking the symmetry of the diagram for complete positivity.
We also note that \emph{Picturing quantum processes}~\cite{CoeckeKissinger17} does not distinguish between diagrams and their evaluation as matrices.
Moreover, their definition of cq-map includes the complete positivity condition, thus they do not give a name to the matrices that we call cq-maps.

There is a functor $\mathtt{double} : \mathbf{Mat}_\C \to \mathbf{CQMap}$ which sends a dimension $n$ to the pair $(1, n)$ and a complex matrix $f$ to its \emph{double} $\widehat{f}$, the cq-map given by tensoring with its conjugate $\widehat{f} = f \otimes f^\star$.
The double of a scalar $s : 1 \to 1$ gives the same result as the Born rule $\hat{s} = s \bar{s} = \vert s \vert^2$.
The double of any matrix $f : m \to n$ automatically satisfies the complete-positivity condition, it satisfies causality iff $f$ is an isometry, i.e. $f^\dagger \fcmp f = \id(n)$.
Furthermore the functor $\mathtt{double}$ is faithful up to a global phase~\cite[Proposition~6.6]{CoeckeKissinger17}.
A cq-map is called \emph{pure} if it's in the image of $\mathtt{double}$ and \emph{mixed} otherwise.

There is also a functor $\mathtt{single} : \mathbf{Mat}_\R \to \mathbf{CQMap}$ which sends a dimension $n$ to the pair $(1, n)$ and a real matrix $f : m \to n$ to itself.
Again, this is always completely positive and it satisfies causality iff $f$ is a stochastic matrix, i.e. all its columns sum to one.
For every dimension $x \in \N$, there are channels $\mathtt{measure}(x) : (1, x) \to (x, 1)$ and $\mathtt{encode}(x) : (x, 1) \to (1, x)$ with underlying matrix given by $\mathtt{spider}_{2, 1}(x)$ and $\mathtt{spider}_{2, 1}(x)$.
For every pair of dimensions $a, b \in \N$, we can also define $\mathtt{discard}(a, b) : (a, b) \to (1, 1)$ and $\mathtt{mixed\_state}(a, b) : (1, 1) \to (a, b)$ with underlying matrix given by $\mathtt{spider}_{1, 0}(a) \otimes \mathtt{cup}(b)$ and its dagger.
Thus, the intuition for the causality condition is that the future cannot signal to the past: if we apply a quantum process then discard the output, we might as well have discarded the input.
Abstractly, causality makes the category $\mathbf{Channel}$ \emph{semicartesian}, i.e. the monoidal unit is a terminal object, discarding is the unique arrow from any object into it.

\begin{python}
{\normalfont Implementation of the category $\mathbf{CQMap}$ with \py{CQ} as objects and \py{Channel} as arrows.}

\begin{minted}{python}
@dataclass
class CQ:
    classical: tuple[int, ...] = ()
    quantum: tuple[int, ...] = ()

    def tensor(self, other: CQ) -> CQ:
        return CQ(self.classical + other.classical, self.quantum + other.quantum)

    def downgrade(self) -> tuple[int]:
        return self.classical + 2 * self.quantum

    __matmul__ = tensor

C, Q = lambda x: CQ(classical=x), lambda x: CQ(quantum=x)

@dataclass
class Channel(Composable, Tensorable):
    inside: Tensor[complex]
    dom: CQ
    cod: CQ

    @staticmethod
    def id(x: CQ) -> Channel: return Channel(x, x, Tensor[complex].id(x.downgrade()))

    def dagger(self) -> Channel: return Channel(self.inside.dagger(), self.cod, self.dom)

    @inductive
    def then(self, other: Channel) -> Channel:
        assert self.cod == other.dom
        return Channel(self.inside >> other.inside, self.dom, other.cod)

    @inductive
    def tensor(self, other: Channel) -> Channel:
        inside = ...  # Given by the diagram above.
        return Channel(inside, self.dom @ other.dom, self.cod @ other.cod)

    @staticmethod
    def double(f: Tensor[complex]) -> Channel:
        return Channel(f @ f.map(lambda x: x.conjugate()), Q(f.dom), Q(f.cod))

    @staticmethod
    def single(f: Tensor[float]) -> Channel:
        return Channel(Tensor[complex](f.inside, f.dom, f.cod), C(f.dom), C(f.cod))

    @staticmethod
    def measure(x: tuple[int, ...]) -> Channel:
        return Channel(Tensor[complex].spider(2, 1, x), Q(x), C(x))

    @staticmethod
    def encode(x: tuple[int, ...]) -> Channel:
        return Channel(Tensor[complex].spider(1, 2, x), C(x), Q(x))

    @staticmethod
    def discard(x: CQ) -> Channel:
        inside = Tensor[complex].spider(1, 0, x.classical)\
            @ Tensor[complex].cups(x.quantum, x.quantum[::-1])
        return Channel(inside, x, CQ())
\end{minted}
\end{python}

The category $\mathbf{CQMap}$ inherits a dagger compact closed structure from that of $\mathbf{Mat}_\C$.
The swaps (cups, caps) for classical-quantum systems $(a, b) \in \N \times \N$ are given by tensoring a single swap (cup, cap) for $a$ and a double swap (cup, cap) for $b$.
It also has sums given by element-wise addition, note however that the functor $\mathtt{double}$ does not preserve sums, i.e. $\widehat{f + f'} \neq \widehat{f} + \widehat{f'}$ in general.
In quantum mechanical terms, this corresponds to the distinction between
quantum superposition and probabilistic mixing.
Crucially, the subcategory $\mathbf{Channel}$ of CPTP maps is \emph{not} compact-closed because caps are not causal.
It does not have sums in general, the sum of two channels is not causal but any convex combination is.

\begin{python}
{\normalfont Implementation of $\mathbf{CQMap}$ as a compact closed category with sums.}

\begin{minted}{python}
CQ.l = CQ.r = property(lambda self: self)

for attr in ("swap", "cups", "caps"):
    def channel_method(left: CQ, right: CQ) -> Channel:
        tensor_method = getattr(Tensor, attr)
        return Channel.single(tensor_method(left.classical, right.classical))\
            @ Channel.double(tensor_method(left.quantum, right.quantum))
    setattr(Channel, attr, channel_method)

def __add__(self, other: Channel) -> Channel:
    assert self.dom == other.dom and self.cod == other.cod
    return Channel(self.inside + other.inside, self.dom, self.cod)

@staticmethod
def zero(dom: CQ, cod: CQ) -> Channel:
    return Channel(Tensor.zero(dom.dowgrade(), cod.downgrade()), dom, cod)

Channel.__add__, Channel.zero = __add__, zero
\end{minted}
\end{python}

We're now ready to define $\mathbf{Circ}$ as a free symmetric monoidal category with a monoidal functor $\mathbf{Circ} \to \mathbf{CQMap}$.
The definition of circuits depends on a choice of \emph{gateset}, i.e. a monoidal signature $\Sigma$ together with a functor $[\![-]\!] : F^S(\Sigma) \to \mathbf{Mat}_\C$ from the free symmetric category of pure circuits as complex matrices.
We assume there is a generating object for each finite-dimensional quantum system $\Sigma_0 = \{ \mathtt{qudit}(n) \}_{n > 1}$ and a box $g \in \Sigma_1$ for each pure quantum process, e.g. unitary gates, preparations (kets) and post-selected measurements (bras).
We say the gateset is universal when the interpretation $[\![-]\!] : F^S(\Sigma) \to \mathbf{Mat}_\C$ is full, i.e. every complex matrix is the interpretation of some pure circuit.

We then define an extended signature $cq(\Sigma) \supset \Sigma$ with objects $cq(\Sigma)_0 = \{ \mathtt{digit}(n) \}_{n > 1} + \{ \mathtt{qudit}(n) \}_{n > 1}$ for classical and quantum systems of each dimension, and boxes given by:
$$cq(\Sigma)_1 = \{ \hat{g} \}_{g \in \Sigma_1}
+ \{ \mathtt{measure}(n) : \mathtt{qudit}(n) \to \mathtt{digit}(n) \}_{n > 1}
+ \{ \mathtt{encode}(n) : \mathtt{digit}(n) \to \mathtt{qudit}(n) \}_{n > 1}$$
Let $\mathbf{Circ} = F^S(cq(\Sigma))$ be the free symmetric category it generates, where the diagrams are called mixed quantum circuits.
Their interpretation $[\![-]\!] : \mathbf{Circ} \to \mathbf{CQMap}$ is given by
$[\![\mathtt{digit}(n)]\!] = (n, 1)$ and $[\![\mathtt{qudit}(n)]\!] = (1, n)$ on objects and
$[\![\hat{g}]\!] = \mathtt{double}([\![g]\!])$.

The evaluation of any mixed circuit is always completely-positive~\cite[Corollary~8.6]{CoeckeKissinger17}.
Let $\mathbf{CausalCirc} \injects \mathbf{Circ}$ be the subcategory of causal processes, i.e. $\mathbf{CausalCirc} = F^S(cq(\Sigma'))$ for $\Sigma' \sub \Sigma$ the set of boxes that are interpreted as isometries.
If the gateset is universal, then the interpretation $\mathbf{CausalCirc} \to \mathbf{Channel}$ is full~\cite[Theorem~8.96]{CoeckeKissinger17}.
More explicitly, every quantum channel $f : (a, b) \to (c, d)$ can be written as:
$$f = \mathtt{encode}(a) \otimes b \otimes \mathtt{double}(\vert 0 \rangle_n)
\s \fcmp \s \mathtt{double}(g) \s \fcmp \s
\mathtt{measure}(c) \otimes d \otimes \mathtt{discard}(n)$$
for some $n$-dimensional ancilla $\vert 0 \rangle_n$ and an $(n a b) \times (n c d)$ unitary matrix $g$.
This gives us a general intuition for what it means to take a shot at a quantum circuit: 1) we prepare some qubits in the zero state, 2) we perform classically-controlled unitary gates, 3) we measure some of the qubits and discard the others.
Drawing digits and qudits as thin and thick wires, encode and measure as spiders, discard as three horizontal lines, we get the following diagram for a generic circuit:
\ctikzfig{img/qnlp/stinespring}

DisCoPy implements \py{Circuit} as a subclass of \py{Diagram} with objects generated by two families of objects \py{Digit(n)} and \py{Qudit(n)} indexed by natural numbers \py{n > 1}, where \py{bit = Ty(Digit(2))} and \py{qubit = Ty(Qudit(2))}.
The class \py{Gate} is a subclass of \py{Box} and \py{Circuit} with an attribute \py{array} which will define its interpretation as a pure circuit, i.e. a unitary matrix.
We also have boxes for \py{Ket} and its dagger \py{Bra}, \py{Measure} and its dagger \py{Encode}, \py{Discard} and its dagger \py{MixedState}.

\begin{python}
{\normalfont Implementation of the category $\mathbf{Circ}$ with \py{Digit} and \py{Qudit} as generating objects and \py{Circuit} as arrows.}

\begin{minted}{python}
class Digit(Ob):
    def __init__(self, n: int):
        self.n = n
        super().__init__(name="bit" if n == 2 else "Digit({})".format(n))

class Qudit(Ob):
    def __init__(self, n: int):
        self.n = n
        super().__init__(name="qubit" if n == 2 else "Qudit({})".format(n))

bit, qubit = Ty(Digit(2)), Ty(Qudit(2))

class Circuit(Diagram): pass

class Gate(Box, Circuit):
    def __init__(self, name: str, dom: Ty, cod: Ty,
                 array: list[list[int]], is_dagger=False):
        self.array = array
        Box.__init__(self, name, dom, cod, is_dagger=is_dagger)

    def dagger(self) -> Gate: return Gate(
        self.name, self.cod, self.dom, self.array, is_dagger=not self.is_dagger)

class Bra(Box, Circuit):
    def __init__(self, *digits: int, base=2):
        self.digits, self.base = digits, base
        name = "Bra({}, base={})".format(', '.join(map(str, digits)), base)
        Box.__init__(self, name, qubit ** len(digits), qubit ** 0)

    def dagger(self) -> Ket: return Ket(*self.digits, base=self.base)

class Ket(Box, Circuit):
    def __init__(self, *digits: int, base=2):
        self.digits, self.base, name = digits, base
        name = "Ket({}, base={})".format(', '.join(map(str, digits)), base)
        Box.__init__(self, name, qubit ** 0, qubit ** len(digits))

    def dagger(self) -> Bra: return Bra(*self.digits, base=self.base)

class Encode(Box, Circuit):
    def __init__(self, dom=bit):
        obj, = dom.inside
        assert isinstance(obj, Digit)
        Box.__init__("Encode({})".format(n), dom, Ty(Qudit(obj.n)))

    def dagger(self) -> Measure: return Measure(self.cod)

class Measure(Box, Circuit):
    def __init__(self, dom=qubit):
        obj, = dom.inside
        assert isinstance(obj, Qudit)
        Box.__init__("Measure({})".format(n), dom, Ty(Digit(obj.n)))

    def dagger(self) -> Encode: return Encode(self.cod)

class Discard(Box, Circuit):
    def __init__(self, x: Ty):
        Box.__init__("Discard({})".format(x), x, Ty())

    def dagger(self) -> MixedState: return MixedState(self.dom)

class MixedState(Box, Circuit):
    def __init__(self, x: Ty):
        Box.__init__("MixedState({})".format(x), Ty(), x)

    def dagger(self) -> Discard: return Discard(self.cod)
\end{minted}
\end{python}

As discussed in example~\ref{example:pivotal-circuit}, the category $\mathbf{Circ}$ also has a dagger compact closed structure where the cups and caps for qudits are given by scaled Bell states and post-selected Bell measurements respectively.
The cups for digits are given by the result of measuring a scaled Bell state, or equivalently as an (unnormalised) correlated probability distributions, the caps can be thought of as classical post-selection on two digits being equal.
We can freely add sums to $\mathbf{Circ}$ and execute (simulate) a formal sum of circuits by executing (simulating) each circuit and adding up the results.
If we can take formal sums, there's no reason not to also take linear combinations of circuits.
Via the Born rule, we can already define positive real scalars as the evaluation of pure quantum gates on zero qubits, such as the $\sqrt 2$ scalars of our first example~\ref{example:circuit-alive-loves-bob}.
What we're missing are \emph{mixed scalars} which get applied after the Born rule.

\begin{python}\label{listing:mixed-scalars}
{\normalfont Implementation of pure and mixed scalars in $\mathbf{Circ}$.}

\begin{minted}{python}
class Sqrt(Gate):
    def __init__(self, x: float):
        super().__init__("$\\sqrt {}$".format(x), Ty(), Ty(), array=[[math.sqrt(x)]])

class Scalar(Box, Circuit):
    def __init__(self, z: complex, is_pure=False):
        self.z, self.is_pure = z, is_pure
        Box.__init__(self, "Scalar({}, is_pure={})".format(z, is_pure), Ty(), Ty())
\end{minted}
\end{python}

\begin{python}
{\normalfont Implementation of the subcategory of $\mathbf{Circ}$ spanned by qubits as a compact closed category.}

\begin{minted}{python}
Digit.l = Digit.r = Qudit.l = Qudit.r = property(lambda self: self)

X, Y, Z, H = (Gate(name, qubit, qubit, array) for name, array in zip("XYZH", [
    [[0, 1], [1, 0]], [[0, -1j], [1j, 0]], [[1, 0], [0, -1]],
    [[1 / sqrt(2), 1 / sqrt(2)], [1 / sqrt(2), -1 / sqrt(2)]]]))
CX = Gate('CX', qubit ** 2, qubit ** 2, [[1, 0, 0, 0],
                                         [0, 1, 0, 0],
                                         [0, 0, 0, 1],
                                         [0, 0, 1, 0]])
@staticmethod
@nesting
def cups(left: Ty, right: Ty):
    if left == right == qubit: return Sqrt(2) @ Ket(0, 0) >> H @ qubit >> CX
    raise NotImplementedError

Circuit.cups, Circuit.caps = cups, lambda left, right: cups(left, right).dagger()
\end{minted}
\end{python}

\py{Circuit} comes with a Boolean property \py{is_pure} which defines the subcategory of pure circuits, i.e. that we can interpret as \py{Tensor}.
The \py{eval} method now comes with an optional Boolean argument \py{mixed}: if \py{not mixed} and \py{is_pure} we apply a \py{Tensor}-valued functor, otherwise a \py{Channel}-valued functor.
The optional \py{backend} argument allow to go from numerical simulation to quantum hardware: it translates a circuit diagram with only \py{Digit} outputs (i.e. all qudits have been measured or discarded) into a \py{pytket.Circuit} before executing it on a quantum device.
As of today, DisCoPy does not implement the execution of \py{Encode} on quantum hardware yet, this would require to decide which quantum gate to perform depending on the result of a previous measurement.
Although simulating qudits makes no real difference than simulating qubits, we can only execute qubit circuits yet.

\begin{python}
{\normalfont Implementation of the \py{Circuit.eval} method.}

\begin{minted}{python}
for cls in (Gate, Bra, Ket):                  setattr(cls, "is_pure", True)
for cls in (Encode, Measure, Discard, Mixed): setattr(cls, "is_pure", False)

Circuit.is_pure = property(lambda self:
    all(box.is_pure for box in self.boxes)
    and all(isinstance(obj, Qudit) for obj in (self.dom @ self.cod).inside))

class PureEval(Functor):
    ob = ar = {}
    dom, cod = Category(Ty, Circuit), Category(tuple[int, ...], Tensor[complex])

    def __call__(self, other):
        if isinstance(other, Qudit): return [other.n]
        if isinstance(other, Gate) and not other.is_dagger:
            return Tensor[complex](other.array, self(other.dom), self(other.cod))
        if isinstance(other, Bra): return self(other.dagger()).dagger()
        if isinstance(other, Ket):
            if not other.digits: return Tensor.id([])
            if len(other.digits) == 1:
                inside = [[i == other.digits[0] for i in range(other.base)]]
                return Tensor[complex](inside, [], [other.base])
            head, *tail = other.digits
            return self(Ket(head, base=other.base)) @ self(Ket(*tail, base=other.base))
        return super().__call__(other)

class MixedEval(Functor):
    ob = ar = {}
    dom, cod = Category(Ty, Circuit), Category(CQ, Channel)

    def __call__(self, other):
        if isinstance(other, Qudit): return Q([other.n])
        if isinstance(other, Digit): return C([other.n])
        if isinstance(other, Scalar): return Channel([[
            abs(other.z) ** 2 if other.is_pure else other.z]], CQ(), CQ())
        if isinstance(box, (Gate, Bra, Ket)): return Channel.double(PureEval()(box))
        if isinstance(box, Encode): return Channel.encode(self(box.dom))
        if isinstance(box, Measure): return Channel.measure(self(box.dom))
        if isinstance(box, Discard): return Channel.discard(self(box.dom))
        if isinstance(box, MixtedState): return self(box.dagger()).dagger()
        return super().__call__(other)

def eval(self, mixed=True, backend=None) -> Tensor | Channel:
    if backend is not None: ...  # Interface with pytket.
    return PureEval()(self) if not mixed and self.is_pure else MixedEval()(self)
\end{minted}
\end{python}

\begin{example}
We can simulate a Bell test experiment by applying the evaluation functor $\mathbf{Circ} \to \mathbf{Channel}$.

\begin{minted}{python}
circuit = Ket(0, 0) >> H @ qubit >> CX >> Measure() @ Measure()
circuit.draw()
\end{minted}
\ctikzfig{img/qnlp/bell-test}
\begin{minted}{python}
assert circuit.eval() == Channel([[.5, 0, 0, .5]], CQ(), C([2, 2]))
\end{minted}
\end{example}
