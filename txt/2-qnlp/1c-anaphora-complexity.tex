%!TEX root = ../../THESIS.tex

\subsection{Anaphora and the quantum complexity of language}\label{section:anaphora}

While the meaning of \emph{lexical words} (also called \emph{content} words) such as nouns and verbs are extracted from text data, DisCoCat models allow to encode \emph{functional words} such as pronouns and conjunctions in terms of the Frobenius algebras, a.k.a. spiders, we discussed in section~\ref{subsection:hypergraph}.
This \emph{Frobenius anatomy of word meanings} was first applied to relative pronouns~\cite{SadrzadehEtAl13,SadrzadehEtAl14}, then to coordination~\cite{Kartsaklis16} as well as intonation~\cite{KartsaklisSadrzadeh15}.
In previous work with Coecke, de Felice and Marsden~\cite{CoeckeEtAl18a} as well as in a subsequent dissertation~\cite{Toumi18a}, we proposed the use of spiders to model \emph{anaphora}, expressions such as personal pronouns whose meaning depends on another expression in context, connecting the diagrams for sentences together into a diagram for \emph{discourse}.
This proposal came with an algorithm for constructing a \emph{relational database} from any such discourse diagram and for translating the diagrams of questions into database queries.

This discourse-to-database translation was refined in later work with de Felice and Meichanetzidis~\cite{FeliceEtAl19} where we defined the \emph{functorial question answering} problem as the application of a given DisCoCat model to a question diagram.
In the case of Boolean-valued models $F : \G \to \mathbf{Mat}_\S$, we proved that this question-answering problem is in fact equivalent to \emph{conjunctive query evaluation}, which is $\mathtt{NP}$-complete by a celebrated theorem of Chandra and Merlin~\cite{ChandraMerlin77}.
Conjunctive queries can be defined as Peircean diagrams with no bubbles, i.e. only spiders and predicate boxes, or equivalently as the first-order logic formulae with existentials and conjunction but no negation, see Bonchi et. al for a diagrammatic treatment~\cite{BonchiEtAl18}.
The Chandra-Merlin theorem is based on the construction of a \emph{canonical model} for the given query, i.e. a canonical functor given a diagram, then reducing evaluation to the problem of graph homomorphism between models.
Once translated in terms of Boolean DisCoCat models, the same result implies that question-answering (i.e. the application of a functor to a diagram) is equivalent to the \emph{entailment problem}: given two sentences, does the truth of one imply that the other?

DisCoCat models with anaphoric spiders are unsatisfying in two opposite ways: they are not expressive enough to encode negation\footnote
{Note that the results of \cite{FeliceEtAl19} assume that the sentence type is mapped to the unit, i.e. the meaning of a sentence is a scalar in $\B$.
Preller~\cite{Preller14a,Preller14} takes an alternative, four-valued approach to first-order logic with pregroups where the sentence type is given dimension $2$: a sentence is either true, false, neither or both.}
but if allow arbitrary anaphora, they are already too expressive to be computed efficiently.
The first point cannot be avoided if we want our models to be tractable: adding negation to conjunctive queries will generate all of first-order logic, for which question-answering (i.e. model checking) is $\mathtt{PSPACE}$-complete and entailment (i.e. the Entscheidungsproblem) undecidable.
We may get around the second dissatisfaction by adding restrictions on anaphoric expressions.
For example, requiring that the corresponding query has \emph{bounded tree-width}~\cite{ChekuriRajaraman00} ensures that question answering is solvable in polynomial time.
This restriction may be motivated in terms of \emph{bounded short-term memory}: a query with tree-width $k$ corresponds to a first-order logic with $k$ variables, each of which may be bound and reused multiple times.
We refer to Abramsky and Shah~\cite{AbramskyShah18} for a comonadic approach to such resource bounds.

If we go from Booleans to natural numbers, we get a \emph{counting problem}: we want to know not only whether but \emph{how many} answers a question has in a given model.
This answer-counting problem is complete for $\mathtt{\#P}$, the generalisation of $\mathtt{NP}$ from decision to counting problems.
By extension, evaluating DisCoCat models over the real or complex numbers (with finite precision) is also $\mathtt{\#P}$-complete.
The closest decision complexity class is $\mathtt{PP}$, also called \emph{Majority-P}, the class of problems solvable in probabilistic polynomial time with no error bounds, which amounts to computing the most significant digit of a $\mathtt{\#P}$ problem.
If we write $\mathtt{P}^\mathtt{X}$ for the class of problems solvable in polynomial time with access to an oracle solving any problem of $\mathtt{X}$ in one step, we can use a binary search to prove $\mathtt{P}^\mathtt{PP} = \mathtt{P}^\mathtt{\#P}$.
One indication for how much harder counting is compared to decision problems is \emph{Toda's theorem}~\cite{Toda91}.
It states that $\mathtt{PH} \sub \mathtt{P}^\mathtt{\#P}$ where $\mathtt{PH}$ is the \emph{polynomial hierarchy}, the union of all towers of $\mathtt{NP}$-oracles, i.e. $\mathtt{PH} = \cup_{n \in \N} \Sigma_n$ where $\Sigma_0 = \mathtt{P}$ and $\Sigma_{n + 1} = \mathtt{NP}^{\Sigma_n}$.
Intuitively, a $\mathtt{PP}$-oracle for counting problems is at least as powerful than any tower of $\mathtt{NP}$-oracles for decision problems.

In a beautifully simple theorem, Aaronson~\cite{Aaronson05} shows that $\mathtt{PP} = \mathtt{PostBQP}$, the class of problems solvable in polynomial time by a quantum computer given \emph{post-selection}, the ability to make the possible necessary and to choose what outcome we get from a quantum measurement\footnote
{Assuming the many-world hypothesis, Aaronson~\cite{Aaronson05} gives a simple method to achieve post-selection: committing suicide if we do not get the desired outcome.
A less brutal method is to keep on trying until we do, in exponential time on average.}.
In one direction, this equality says that the evaluation of a post-selected quantum circuit can be reduced to tensor contraction: quantum gates, bras and kets are nodes, the qubits connecting them are edges.
In the other, it means that we can use post-selected quantum computation to contract any tensor network, or equivalently to evaluate any monoidal functor from a free compact closed category into $\mathbf{Mat}_\C$.
The related counting class $\mathtt{\#P}$ was originally introduced by Valiant~\cite{Valiant79} to show the completeness the \emph{matrix permanent}.
Aaronson and Arkhipov~\cite{AaronsonArkhipov11} then related it to the complexity of \emph{boson sampling}, a restricted model of quantum computation which they prove cannot be simulated classically unless the \emph{polynomial hierarchy collapses to the third level}.
Although this is less unlikely than $\mathtt{P} = \mathtt{NP}$, i.e. a collapse at level zero, this is still believed to be a strong indication that quantum computers cannot be efficiently classically simulated.

Removing post-selection from $\mathtt{PostBQP}$ we get $\mathtt{BQP}$, bounded-error quantum polynomial time, arguably the largest class of decision problems that a physical machine\footnote
{By a physical machine we mean a machine that obeys the laws of quantum mechanics.
This excludes machines exploiting features of general relativity such as closed time-like curves (CTCs).
Using the CTCs of Deutsch~\cite{Deutsch91} a quantum computer can solve all of $\mathtt{PSPACE}$ in polynomial time, while the more restricted CTCs of Lloyd et al.~\cite{LloydEtAl11,LloydEtAl11a} solve all of $\mathtt{PostBQP}$ in polynomial time.
See Pinzani, Gogioso and Coecke~\cite{PinzaniEtAl19} for a diagrammatic treatment of time-travel in terms of traced categories.}
can solve efficiently.
Its classical analog $\mathtt{BPP}$ (bounded-error probabilistic polynomial time) is contained in $\mathtt{BQP}$ because quantum computers can simulate classical ones efficiently, but whether the containment is strict is an open question.
The best we can do is define $\mathtt{BQP}$-complete problems with \emph{circuit approximation} as the canonical example: given the description of a quantum circuit, decide whether measuring the first qubit yields a one, with the promise that the probability for this is bounded away from a half.
Arad and Landau~\cite{AradLandau10} reformulate this in terms of the additive approximation of tensor networks, i.e. the additive approximation of a monoidal functor into $\mathbf{Mat}_\C$.
In one direction, their reduction tells us that we can approximate tensor networks efficiently with a quantum computer.
In the other, it means that if we could approximate such functors with a classical computer then we could also simulate any quantum circuit efficiently.

What do these complexity results imply for evaluating DisCoCat models on quantum computers?
First, that we cannot hope to evaluate them exactly unless we discover (safe and efficient) time travel and prove $\mathtt{PostBQP} = \mathtt{BQP}$.
Second, that we can evaluate them efficiently with a quantum computer, up to additive approximation.
Third, that if we could do the same with classical computers then they would turn out to be as powerful as quantum computers after all.
We would automatically get a classical Shor algorithm that can outperform the best number sieves mathematicans have come up with, and a classical Grover algorithm that can find a needle in a haystack.
Thus, by interpreting pregroup grammars in terms of tensor networks, DisCoCat models provide a way to reformulate quantum computing in terms of natural language processing.
In short, if $\mathtt{BPP} \neq \mathtt{BQP}$ then quantum computers would allow us to (approximately) answer exponentially bigger natural language questions than classical computers can.
The idea of using quantum circuits to evaluate DisCoCat models was first introduced by Zeng and Coecke~\cite{ZengCoecke16}, where they show a quadratic advantage on a more restricted classification task using Grover's algorithm as subroutine.
Given the size of classical NLP models today, empirical evidence of quantum advantage for natural language processing will most probably require fault-tolerant quantum computers with millions of qubits, if not billions.
In the meantime, we are left to explore the possibilities offered by the small noisy quantum computers of today.
