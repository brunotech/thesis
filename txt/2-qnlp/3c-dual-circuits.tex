%!TEX root = ../../THESIS.tex

\subsection{Differentiating quantum circuits}\label{3-dual-circuits}

In this section, we extend diagrammatic differentiation to the category $\mathbf{Circ}$ of mixed quantum circuits as defined in section~\ref{section:mixed-circuits}.
Recall that the definition of $\mathbf{Circ}$ depends on a choice of gateset, here we assume that this gateset is interpreted as complex matrices parameterised by some number $n \in \N$ of real variables.
That is, we fix an interpretation $[\![-]\!] : \mathbf{Circ} \to \mathbf{Mat}_\S$ for $\S = \R^n \to \C$.
In this context, diagrammatic derivations correspond to the notion of gradient
recipe for parametrised quantum gates introduced by Schuld et al.~\cite{SchuldEtAl19}.

Let $\mathbf{Circ}^+$ be the category with formal sums of mixed quantum circuits as arrows, i.e. the free commutative-monoid-enrichment of $\mathbf{Circ}$.
Again, we want to find a diagrammatic derivation
$\partial : \mathbf{Circ}^+ \to \D[\mathbf{Circ}^+]$
which commutes with the interpretation, i.e. such that
$[\![\partial \hat{f}]\!] = \partial [\![\hat{f}]\!] =
\partial \big( \overline{[\![f]\!]} \otimes [\![f]\!] \big)$
for all circuits $f \in \mathbf{Circ}$.
Note that a diagrammatic derivation for the interpretation of pure quantum circuits does not in general lift to one for mixed quantum circuits.
Indeed, using the product rule we get
$\partial \big( \overline{[\![f]\!]} \otimes [\![f]\!] \big)
\s = \s \partial \overline{[\![f]\!]} \otimes [\![f]\!]
\ + \ \overline{[\![f]\!]} \otimes \partial [\![f]\!]
\s \neq \s \overline{[\![\partial f]\!]} \otimes [\![\partial f]\!]$.

Hence we need equations, called gradient recipes, to rewrite the gradient of a
pure map $\partial [\![\hat{f}]\!]$ as the pure map of a gradient
$[\![\partial \hat{f}]\!]$.
In the special case of Hermitian operators with at most two unique eigenvalues,
gradient recipes are given by the parameter-shift rule. In the general case
where the parameter-shift rule does not apply, gradient recipes require the
introduction of an ancilla qubit.

\begin{theorem}[\cite{SchuldEtAl19}]
For a one-parameter unitary group $f$ with
$[\![f(\theta)]\!] = \exp (i \theta H)$, if $H$ has at most two eigenvalues
$\pm r$, then there is a shift $s \in [0, 2 \pi)$ such that
$[\![r\big(f(\theta + s) - f(\theta - s)\big)]\!] = \partial [\![f(\theta)]\!]$.
\end{theorem}

\begin{corollary}
Mixed quantum circuits with parametrised ZX diagrams as gateset admit diagrammatic differentiation.
\end{corollary}

\begin{proof}
The $Z$ rotation has eigenvalues $\pm 1$, hence the spiders with two legs have
diagrammatic differentiation given by the parameter-shift rule:

\ctikzfig{img/diag-diff/3-2-param-shift}

As for theorem~\ref{theorem-zx-diag-diff}, this extends to
arbitrary-many legs using spider fusion.
\end{proof}

\begin{remark}
In order to encode the substraction of the parameter shift-rule diagrammatically, we
need either to consider formal sums with minus signs (a.k.a. enrichment in
abelian groups) or simply to extend the signature with the $-1$ scalar.
Such \emph{mixed scalars} are implemented in listing~\ref{listing:mixed-scalars}.

\begin{minted}{python}
Circuit.__sub__ = lambda self, other: self + Scalar(-1) @ other
\end{minted}
\end{remark}

\begin{example}
The quantum enhanced feature spaces of Havlicek et al.~\cite{HavlicekEtAl19} are parametrised
classical-quantum circuits.
The quantum classifier can be drawn as a diagram:

\ctikzfig{img/diag-diff/3-3-quantum-enhanced}

where $U(\vec{x})$ depends on the input, $W(\vec{\theta})$ depends on the
trainable parameters and $f$ is a fixed Boolean function encoded as a linear map.
\end{example}

\begin{example}
We can define a class for a parameterised quantum gate and override the default \py{grad} with its gradient recipe.

\begin{minted}{python}
class Rz(Gate):
    def __init__(self, phase: sympy.Expr):
        self.phase = phase
        half_theta = sympy.pi * phase
        array = [[sympy.exp(-1j * half_theta), 0],
                 [0, sympy.exp(1j * half_theta)]]
        super().__init__("Rz({})".format(phase), qubit, qubit, array)

    def grad(self):
        s = Scalar(sympy.pi * self.phase.diff(var))
        return s @ Rz(self.phase + .25) - s @ Rz(self.phase - .25))

phi = sympy.Symbol('\\phi')
circuit = Ket(0, 0) >> Rz(phi + 1) @ Rz(2 * phi - .5) >> Measure() @ Measure()

assert circuit.grad(phi).eval() == circuit.eval().grad(phi)
\end{minted}
\end{example}
