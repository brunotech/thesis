%!TEX root = ../../THESIS.tex

\subsection{Dual diagrams}\label{2-dual-diagrams}

Dual numbers were first introduced by Clifford~\cite{Clifford73}, they are a fundamental tool for \emph{automatic differentiation}~\cite{Hoffmann16}, i.e. they allow to compute the derivative of a function automatically from its definition.

Given a commutative rig $\S$, the rig of dual numbers $\D[\S]$ extends $\S$ by adjoining a new element $\epsilon$ such that $\epsilon^2 = 0$.
Abstractly, $\D[\S] = \S[x] / x^2$ is a quotient of the rig of polynomials with coefficients in $\S$.
Concretely, elements of $\D[\S]$ are formal sums $s + s' \epsilon$ where $s$ and $s'$ are scalars in $\S$.
We write $\pi_0, \pi_1 : \D[\S] \to \S$
for the projection on the real and epsilon component respectively.
Addition and multiplication of dual numbers are given by:
\begin{align} \begin{split}\label{linearity}
(a + a' \ \epsilon ) + (b + b' \ \epsilon)
\quad &= \quad (a + b) \s + \s (a + b') \ \epsilon
\end{split}\\
\begin{split}\label{product-rule}
(a + a' \ \epsilon ) \times (b + b' \ \epsilon)
\quad &= \quad (a \times b) \s + \s (a \times b' \ + \ a' \times b) \ \epsilon
\end{split}
\end{align}

A related notion is that of \emph{differential rig}: a rig $\S$ equipped with a derivation, i.e. a map $\partial : \S \to \S$ which preserves sums and satisfies the Leibniz product rule
$\partial(f \times g) = f \times \partial(g) + \partial(f) \times g$ for all $f, g \in \S$.
An equivalent condition is that the map $f \mapsto f + (\partial f) \epsilon$ is a homomorphism of rigs $\S \to \D[\S]$.
The correspondance also works the other way around: given a homorphism $\partial : \S \to \D[\S]$ such that $\pi_0 \circ \partial = \id_\S$, projecting on the epsilon component is a derivation $\pi_1 \circ \partial : \S \to \S$.
The motivating example is the rig of smooth functions $\S = \R \to \R$, where differentiation is a derivation.
Concretely, we can extend any smooth function $f : \R \to \R$ to a function $f : \D[\R] \to \D[\R]$ over the dual numbers defined by:
\begin{equation}\label{dual-numbers-eq}
f(a + a' \epsilon) \quad = \quad f(a) \s + \s a' \times (\partial f)(a) \epsilon
\end{equation}

We can use equations~\ref{linearity}, \ref{product-rule} and \ref{dual-numbers-eq} to derive the usual rules for gradients in terms of dual numbers.
For the identity function we have $\id(a + a' \epsilon) = \id(a) + a' \epsilon$, i.e. $\partial \id = 1$.
For the constant functions we have $c(a + a' \epsilon) = c(a) + 0 \epsilon$, i.e. $\partial c = 0$.
For addition, multiplication and composition of functions, we can derive the following \emph{linearity}, \emph{product} and \emph{chain} rules:
\begin{align} \begin{split}
    (f + g)(a + a' \epsilon)
    \s &= \s (f + g)(a) \s + \s a' \times (\partial f + \partial g)(a) \epsilon
\end{split}\\ \begin{split}
    (f \times g)(a + a' \epsilon)
    \s &= \s (f \times g)(a) \s + \s a' \times (f \times \partial g \ + \ \partial f \times g)(a) \epsilon
\end{split}\\ \begin{split}
    (f \circ g)(a + a' \epsilon)
    \s &= \s (f \circ g)(a) \s + \s a' \times (\partial g \ \times \ \partial f \circ g)(a) \epsilon
\end{split} \end{align}

This generalises to smooth functions $\R^n \to \R^m$, where the partial derivative $\partial_i$ is a derivation for each $i < n$.
The functions $\F_2^n \to \F_2^m$ on the two-element field $\F_2$ with elementwise XOR as sum and conjunction as product also forms a differential rig.
The partial derivative is given by $(\partial_i f)(\vec{x}) = f(\vec{x}_{[x_i \mapsto 0]}) \oplus f(\vec{x}_{[x_i \mapsto 1]})$.
Intuitively, the $\F_2$ gradient $\partial_i f(\vec{x}) \in \F_2^m$ encodes which coordinates of $f(\vec{x})$ actually depend on the input $x_i$.
An example of differential rig that isn't also a ring is given by the set $\N[X]$ of polynomials with natural number coefficients, again each partial derivative is a derivation.

A more exotic example is the rig of Boolean functions with elementwise disjunction as sum and conjunction as product.
Boolean functions $\B^n \to \B^m$ can be represented as tuples of $m$ propositional formulae over $n$ variables.
The partial derivative $\partial_i$ for $i < n$ is defined by induction over the formulae:
for variables we have $\partial_i x_j = \delta_{ij}$, for constants $\partial_i 0 = \partial_i 1 = 0$ and for negation $\partial_i \neg \phi = \neg \partial_i \phi$. The derivative of disjunctions and conjunctions are given by the linerarity and product rules.
Equivalently, the gradient of a propositional formula can be given by $\partial_i \phi = \neg \phi_{[x_i \mapsto 0]} \land \phi_{[x_i \mapsto 1]}$.
Concretely, a model satisfies $\partial_i \phi$ if and only if it satisfies $\phi \leftrightarrow x_i$: the derivative is true when the variable and the formula are positively correlated.
Substituting $x_i$ with its negation, we get that a model satisfies $\partial_i \phi_{[x_i \mapsto \neg x_i]}$ if and only if it satisfies $\phi \leftrightarrow \neg x_i$, i.e. iff variable and formula are anti-correlated.
Note that although $\B$ and $\F_2$ are isomorphic as sets, they are distinct rigs.
Their derivations are related however by $\partial^{\F_2}_i f \mapsto \partial^\B_i \phi \lor \partial^\B_i \phi_{[x_i \mapsto \neg x_i]}$ for $\phi : \B^n \to \B$ the formula corresponding to the function $f : \F_2^n \to \F_2$.
That is, a Boolean function depends on an input variable precisely when either the corresponding formula is positively correlated or anti-correlated.

Our main technical contribution is to generalise dual numbers and derivations from rigs to monoidal categories with sums.
Given a monoidal category $C$ with sums (i.e. enriched in commutative monoids), we define the category $\D[C]$ by adjoining a scalar (i.e. an endomorphism of the monoidal unit) $\epsilon$ and quotienting by\footnote{
In the case when $C$ is not braided we also require the axiom $\epsilon \otimes f = f \otimes \epsilon$.
} $\epsilon \otimes \epsilon = 0$.
Concretely, the objects of $\D[C]$ are the same as those of $C$, the arrows
are given by formal sums $f + f' \epsilon$ of parallel arrows $f, f' \in C$.
Composition and tensor are both given by the product rule:
\begin{align}
    (f + f' \epsilon) \ \fcmp \ (g + g' \epsilon)
    &\s = \s f \ \fcmp \ g \s + \s (f' \fcmp g \ + \ f \fcmp g') \ \epsilon\\
    (f + f' \epsilon) \otimes (g + g' \epsilon)
    &\s = \s f \otimes g \s + \s (f' \otimes g \ + \ f \otimes g') \ \epsilon
\end{align}

We say that a unary operator on homsets $\partial : \coprod_{x,y} C(x, y) \to C(x, y)$ is a derivation whenever it satisfies the product rules for both composition
$\partial (f \fcmp g) = (\partial f) \fcmp g + f \fcmp (\partial g)$ and tensor
$\partial (f \otimes g) = (\partial f) \otimes g + f \otimes (\partial g)$.
An equivalent condition is that the map $f \mapsto f + (\partial f) \epsilon$ is a sum-preserving monoidal functor $C \to \D[C]$.
Again, the correspondance between dual numbers and derivations works the other way around: given a sum-preserving monoidal functor $\partial : C \to \D[C]$ such that
$\pi_0 \circ \partial = \id_{C}$, projecting on the epsilon component gives a derivation $\pi_1 \circ \partial : \coprod_{x,y} C(x, y) \to C(x, y)$.
The following propositions characterise the derivations on the category of matrices valued in a commutative rig $\S$.

\begin{proposition}
Dual matrices are matrices of dual numbers, i.e. we have $\D[\mathbf{Mat}_\S] \simeq \mathbf{Mat}_{\D[\S]}$ for all commutative rigs $\S$.
\end{proposition}

\begin{proof}
The isomorphism is given by
$$\big( \sum_{ij} f_{ij} \ket{j} \bra{i} \big)
\ + \ \big( f'_{ij} \sum_{ij} \ket{j}  \bra{i} \big) \epsilon
\s \longleftrightarrow \s
\sum_{ij} (f_{ij} + f'_{ij} \epsilon) \ket{j} \bra{i}$$
\end{proof}

\begin{proposition}
Derivations on $\mathbf{Mat}_\S$ are in one-to-one correspondance with
derivations on $\S$.
\end{proposition}

\begin{proof}
A derivation on $\mathbf{Mat}_\S$ is uniquely determined by its action on
scalars in $\S$. Conversely, applying a derivation $\partial : \S \to \S$
entrywise on matrices yields a derivation on $\mathbf{Mat}_\S$.
\end{proof}

DisCoPy implements parameterised matrices with SymPy~\cite{MeurerEtAl17} expressions as entries.
The method \py{Tensor.grad} takes a SymPy variable and applies element-wise symbolic differentiation.

\begin{python}
{\normalfont Implementation of gradients of parameterised tensors with SymPy.}

\begin{minted}{python}
def grad(self: Tensor[sympy.Expr], var: sympy.Symbol):
    return self.map(lambda x: x.diff(var))

Tensor.grad = grad
\end{minted}
\end{python}

\begin{example}
We can check the product rule for tensor and composition of matrices.

\begin{minted}{python}
x = sympy.Symbol('x')
f = Tensor([[x + 1, 2 * x], [x ** 2, 1 / x     ]], [2], [2])
g = Tensor([[1,     2],     [2 * x, -1 / x ** 2]], [2], [2])

assert f.grad(x) == g
assert (f @ g).grad(x)  == f.grad(x) @ g  +  f @ g.grad(x)
assert (f >> g).grad(x) == f.grad(x) >> g + f >> g.grad(x)
\end{minted}
\end{example}

Fix a monoidal signature $\Sigma$ and let $C_\Sigma^+$ be the free monoidal category with sums that it generates, i.e. arrows are formal sums of diagrams as defined in sections~\ref{subsection:dagger-sums-bubbles} and \ref{subsection:monoidal-daggers-sums-bubbles}.
We assume our diagrams are interpreted as matrices, i.e. we fix a sum-preserving monoidal functor $[\![-]\!]  : C_\Sigma^+ \to \mathbf{Mat}_\S$ for $\S$ a commutative rig with a derivation $\partial : \S \to \S$.
Our main two examples are the ZX-calculus of Coecke and Duncan~\cite{CoeckeDuncan08} with smooth functions $\R^n \to \R$ as phases and the algebraic ZX-calculus over $\S$, introduced by Wang~\cite{Wang20}.
Applying the dual number construction to $C_\Sigma^+$, we get the category of \emph{dual diagrams} $\D[C_\Sigma^+]$ which is where diagrammatic differentiation happens.
By the universal property of $C_\Sigma^+$, every derivation $\partial : C_\Sigma^+ \to \D[C_\Sigma^+]$ is uniquely determined by its image on the generating boxes in $\Sigma_1$.
Intuitively, if we're given the derivative for each box, we can compute the derivative for every sum of diagram using the product rule.

We say that the interpretation $[\![-]\!] : C_\Sigma^+ \to \mathbf{Mat}_\S$ \emph{admits diagrammatic differentiation} if there is a derivation $\partial$ on $C_\Sigma^+$ such that $[\![-]\!] \circ \partial = \partial \circ [\![-]\!]$.
That is, the interpretation of the gradient $[\![\partial d]\!]$ coincides with the gradient of the interpretation $\partial [\![d]\!]$ for all sums of diagrams $d \in C_\Sigma^+$.
We depict the gradient $\partial d$ as a bubble surrounding the diagram $d$, as discussed in sections~\ref{subsection:dagger-sums-bubbles} and \ref{subsection:monoidal-daggers-sums-bubbles}.
Once translated to string diagrams, the axioms for derivations on monoidal
categories with sums become:
\ctikzfig{img/diag-diff/2-1a-product-rule}
\ctikzfig{img/diag-diff/2-1b-product-rule}

We implement dual diagrams with a method \py{Diagram.grad} which takes SymPy variables and returns formal sums of diagrams by applying the product rules for tensor and composition.
By default, we assume that boxes are constant, i.e. their gradient is the empty sum.

\begin{python}
{\normalfont Implementation of dual diagrams.}

\begin{minted}{python}
Box.grad = lambda self, var: Sum([], self.dom, self.cod)

def grad(self: Diagram, var: sympy.Symbol):
    if len(self) == 0: return Sum([], self.dom, self.cod)
    left, box, right = self.layers[0]
    return left @ box.grad(var) @ right >> self[1:]\
        + left @ box @ right >> self[1:].grad(var)

Diagram.grad = grad
\end{minted}
\end{python}

\begin{example}
We can override the default \py{grad} method and check the product rule for diagrams.

\begin{minted}{python}
class DataBox(Box):
    def __init__(self, name: str, dom: Ty, cod: Ty, data: sympy.Expr):
        self.data = data
        super().__init__(name, dom, cod)

    def __eq__(self, other):
        if not isinstance(other, DataBox): return super().__eq__(other)
        return super().__eq__(other) and self.data == other.data

    def grad(self, var):
        return DataBox(self.name, self.dom, self.cod, self.data.diff(var))

phi = sympy.Symbol('\\phi')
x, y, z = map(Ty, "xyz")
f, g = DataBox('f', x, y, phi ** 2), DataBox('g', y, z, 1 / phi)

assert (f @ g).grad(phi) == f.grad(phi) @ g + f @ g.grad(phi)
assert (f >> g).grad(phi) == f.grad(phi) >> g + f >> g.grad(phi)
\end{minted}
\end{example}
