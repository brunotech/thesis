%!TEX root = ../../THESIS.tex

\chapter{DisCoPy: Python for the applied category theorist}\label{chapter:discopy}

Python has become the programming language of choice for most applications in both natural language processing (e.g. Stanford NLP~\cite{ManningEtAl14}, NLTK~\cite{LoperBird02} and SpaCy~\cite{HonnibalMontani17}) and quantum computing (with development kits like Qiskit~\cite{Cross18} and PennyLane~\cite{BergholmEtAl20} and interfaces to compilers like pytket~\cite{SivarajahEtAl20}).
Thus, it was the obvious choice of language for an implementation of QNLP.
However, unlike functional programming languages like Haskell, Python has little support for category theory.
Indeed, before the release of DisCoPy, the only existing Python framework for category theory was a module of SymPy~\cite{MeurerEtAl17} that can draw commutative diagrams in finite categories.
Hence, the first step in implementing QNLP was to develop our own framework for applied category theory in Python: DisCoPy.
Its main feature are the drawing of string diagrams (e.g. the grammatical structure of sentences) and the application of functors (e.g. to quantum circuits, either executed on quantum hardware or classically simulated).

String diagrams have become the lingua franca of applied category theory.
However, the definitions one can find in the literature usually fall into one of two extremes: either definitions by general abstract nonsense or definitions by example and appeal to intuition.
On one side of the spectrum, the standard technical reference has become the \emph{Geometry of tensor calculus}~\cite{JoyalStreet91} where Joyal and Street define string diagrams as equivalence classes of labeled topological graphs embedded in the plane and then characterise them as the arrows of free monoidal categories.
On the other, \emph{Picturing quantum processes}~\cite{CoeckeKissinger17} contains over a thousand string diagrams but their formal definition as well as any mention of category theory are relegated to mere appendices.

The aims of this chapter are three-fold: 1) it gives an overview of the DisCoPy package and its design principles, 2) it introduces elementary category theory to the Python programmer and 3) it introduces object-oriented programming to the applied category theorist.
The first section introduces categories and functors with no mathematical prerequisites apart from sets and monoids.
The second section introduces monoidal categories, defining string diagrams from first principles.
The third section defines the drawing and reading algorithms for string diagrams, which arise as the two sides of the equivalence between the premonoidal and the topological definitions.
The fourth section introduces monoidal categories with extra structure and the inheritence mechanism which implements this hierarchy of structure.
The fifth section gives the category theoretic foundations for our definition of diagrams, which we call the premonoidal approach, it discusses the relationship between this approach and the exisiting graph-based data structures for diagrams in symmetric monoidal categories.

\paragraph{Too long; didn't read.}
We provide a brief summary for the reader who wishes to skip the category theory of this chapter and go straight to the QNLP of chapter~\ref{chapter-2:qnlp}.
String diagrams are defined with respect to a \emph{monoidal signature} $\Sigma$: a set of \emph{objects} $\Sigma_0$, a set of \emph{boxes} $\Sigma_1$ and a pair of functions $\dom, \cod : \Sigma_1 \to \Sigma_0^\star$ assigning input and output \emph{types} (i.e. lists of objects) to each box.
A \emph{layer} is a triple $(u, f, v) \in L(\Sigma) = \Sigma_0^\star \times \Sigma_1 \times \Sigma_0^\star$ of a box $f \in \Sigma_1$ with types $u \in \Sigma_0^\star$ on the left and $v \in \Sigma_0^\star$ on the right, where we define $\dom(x, f, y) = x \dom(f) y$ and $\cod(x, f, y) = x \cod(f) y$.
Finally, a \emph{diagram} $d$ is given by a domain $\dom(d) \in \Sigma_0^\star$ and a list of layers $d_1, \dots, d_n \in L(\Sigma)$ such that $\dom(d_1) = \dom(d)$ and $\dom(d_{i + 1}) = \cod(d_i)$ for $i \leq n$.

Given a type $x \in \Sigma_0^\star$, we define the \emph{identity} diagram $\id(x)$ with $\dom(\id(x)) = x$ and an empty list of layers.
Given two diagrams $d$ and $d'$ with $\cod(d) = \dom(d')$, we define their \emph{composition} $d \fcmp d'$ by concatenating of their layers.
Given a diagram $d$ and a type $x \in \Sigma_0^\star$, we define the left and right \emph{whiskering} $x \otimes d$ and $d \otimes x$ by concatenating $x$ to the left and right of each layer in $d$.
Every diagram can be written in terms of boxes, identity, composition and whiskering.
Given two diagrams $d$ and $d'$, we can define their \emph{tensor} as $d \otimes d' = d \otimes \dom(d') \ \fcmp \ \cod(d) \otimes d'$.
The \emph{interchanger} relation induced by $d \otimes \dom(d') \ \fcmp \ \cod(d) \otimes d' \s \sim \s \dom(d) \otimes d' \ \fcmp \ d \otimes \cod(d')$ relates this biased definition of tensor to the one in the opposite direction.

A \emph{monoidal category}\footnote
{What we call a monoidal category technically is a \emph{coloured PRO}, see remark~\ref{remark:coloured-PRO}.} is a monoidal signature $C$ together with an identity function $\id : C_0 \to C_1$, a (partial) composition function $\then : C_1 \times C_1 \to C_1$ and a tensor function $\tensor : C_1 \times C_1 \to C_1$ subject to some axioms spelled out in section~\ref{section:monoidal}.
Diagrams up to interchanger are the \emph{free monoidal category} $C_\Sigma$ generated by the signature $\Sigma$ (see section~\ref{subsection:free-monoidal}).
In practice, this means that we can define a \emph{monoidal functor} $F : C_\Sigma \to D$ as a \emph{signature homorphism} $F : \Sigma \to D$, i.e. a pair of functions $F_0 : \Sigma_0 \to D_0$ and $F_1 : \Sigma_1 \to D_1$ compatible with $\dom$ and $\cod$.
Intuitively, once we have specified the interpretation of each object and each box, the interpretation of every diagram is fixed.
If we take $\Sigma$ to encode the rules of our grammar and $D$ to be a monoidal category of quantum circuits, we get our definition of QNLP models: they are monoidal functors $F : C_\Sigma \to D$.
