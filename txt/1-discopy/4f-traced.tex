%!TEX root = ../../THESIS.tex

\subsection{Traced categories} \label{subsection:traced}

The category $\mathbf{Pyth}$ has another categorical structure in common with hypergraph categories like $\mathbf{Tensor}$: they are both \emph{traced symmetric categories}.
A symmetric category is traced when it comes equipped with a family of functions $\mathtt{trace}_x(a, b) : C(a \otimes x, b \otimes x) \to C(a, b)$ subject to axioms that formalise the intuition that we can trace a morphism $f : a \otimes x \to b \otimes x$ by connecting its input and output $x$-wires in a loop, see Joyal et al.~\cite{JoyalEtAl96}.

Every compact closed category has a trace given by cups and caps, traced symmetric categories allow to express recursion and fixed points more generally in non-rigid categories.
Indeed in the case of a cartesian category such as $\mathbf{Pyth}$, the trace can equivalently be given in terms of \emph{fixed point operators} $\mathtt{fix}_x : C(a \times x, x) \to C(a, x)$~\cite[Proposition~6.8]{Selinger10}.
Dually, in a cocartesian category the trace can be defined in terms \emph{iteration operators} $\mathtt{iter}_x : C(x, a + x) \to C(x, a)$.
When the category has biproducts, it is sufficient to define a \emph{repetition operator} $\mathtt{repeat}_x : C(x, x) \to C(x, x)$~\cite[Proposition~6.11]{Selinger10}.
In the category of finite sets and relations $\mathbf{Mat}_\B$ with the direct sum as tensor, this coincides with the usual notion of reflexive transitive closure~\cite[Proposition~6.3]{JoyalEtAl96}.

\begin{python}
{\normalfont Implementation of the syntax for free traced categories.}

\begin{minted}{python}
class Trace(Box):
    def __init__(self, diagram: Diagram, n=1):
        assert diagram.dom[-n:] == diagram.cod[-n:]
        self.diagram, name = inside, "Trace({}, {})".format(diagram, n)
        super().__init__(name, diagram.dom[:-n], diagram.cod[:-n])

class Diagram(monoidal.Diagram):
    trace = Trace

class Functor(monoidal.Functor):
    dom = cod = Category(Ty, Diagram)

    def __call__(self, other):
        if isinstance(other, Trace):
            n = len(self(other.diagram.dom)) - len(self(other.dom))
            return self.cod.ar.trace(self(other.diagram), n)
        return super().__call__(other)
\end{minted}
\end{python}

\begin{example}
We can draw traced diagrams by applying a traced functor into compact-closed categories with bubbles.

\begin{minted}{python}
def compact_trace(self, n=1):
    assert self.dom[-n:] == self.cod[-n:]
    return self.dom[:-n] @ self.caps(self.dom[-n:], self.dom[-n:].r)\
        >> self @ self.dom[:n].r\
        >> self.cod[:-n] @ self.cups(self.cod[-n:], self.cod[-n:].r)

compact.Diagram.trace = compact_trace

class TracedDrawing(compact.Diagram):
    trace = lambda self, n: compact_trace(self, n).bubble()

Draw = Functor(lambda x: x, lambda f: f, cod=Category(Ty, TracedDrawing))
Diagram.draw = lambda self, **params: Draw(self).draw(**params)

a, b, x = map(Ty, "abx")
Box('f', a @ x, b @ x).draw()
\end{minted}
\ctikzfig{img/closed/trace}
\end{example}

\begin{python}\label{listing:traced-python}
{\normalfont Implementation of $\mathbf{Pyth}$ as a traced cartesian category.}

\begin{minted}{python}
class Function:
    ...
    def iter(self, n=1):
        assert self.is_additive and self.cod[-n:] == self.dom
        dom, cod = self.dom, self.cod[:-n]
        def inside(x):
            i, y = self.inside(x)
            return y if i < len(self.cod) - n else inside(y)
        return Function(inside, dom, cod)

    def fix(self, n=1):
        assert not self.is_additive and self.dom[-n:] == self.cod
        dom, cod = self.dom[:-n], self.cod
        def inside(*xs, ys=None):
            result = self.inside(*xs + (ys or []))
            return ys if result == ys else inside(*xs, ys=result)
        return Function(inside, dom, cod)

    def trace(self, n=1):
        assert self.dom[-n:] == self.cod[-n:]
        dom, cod, traced = self.dom[:-n], self.cod[:-n], self.dom[-n:]
        if self.is_additive:
            iterated = (self.unit(cod) @ traced >> self).iter()
            return dom @ self.unit(traced) >> self >> cod @ iterated >> self.merge(cod)
        fixed = (self >> self.discard(cod) @ traced).fix()
        return self.copy(dom) >> dom @ fixed >> self >> cod @ self.discard(traced)
\end{minted}
\end{python}

\begin{example}
We can compute the golden ratio as a fixed point.
Note that in order to find a fixed point we need a default value to start from.

\begin{minted}{python}
phi = Function(lambda x=1: 1 + 1 / x, [int], [int]).fix()
assert phi() == (1 + sqrt(5)) / 2
\end{minted}

We can define \py{even} as a recursive function.

\begin{minted}{python}
inside = lambda n: (0, n == 0) if n < 2 else (1, n - 2)
even = Function.additive(inside, dom=[int], cod=[bool, int]).iter()
assert even(42)
\end{minted}
\end{example}

\begin{python}\label{listing:traced-matrix}
{\normalfont Implementation of $\mathbf{Mat}_\S$ as a traced biproduct category.}

\begin{minted}{python}
class Matrix:
    ...
    def repeat(self):
        assert self.dtype is bool and self.dom == self.cod
        return sum(Matrix.id(self.dom).then(*n * [self]) for n in range(self.dom + 1))

    def trace(self, n=1):
        assert self.dtype is bool
        A, B, C, D = (row >> self >> column
                      for row in [self.id(self.dom - n) @ self.unit(n),
                                  self.unit(self.dom - n) @ self.id(n)],
                      for column in [self.id(self.cod - n) @ self.discard(n),
                                     self.discard(self.cod - n) @ self.id(n)])
        return A + (B >> D.repeat() >> C)
\end{minted}
\end{python}
