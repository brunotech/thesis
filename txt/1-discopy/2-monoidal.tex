%!TEX root = ../../THESIS.tex

\section{String diagrams in Python}

In the previous section, we introduced the idea of arrows in free categories as formal data pipelines and functor application as their evaluation in concrete categories such as $\mathbf{Pyth}$, $\mathbf{Mat}$ or $\mathbf{Circ}$ where the computation happens.
For now, our pipelines are rather basic because they are linear: we cannot express functions of multiple arguments, nor tensors of order higher than 2, nor circuits with multiple qubits in any explicit way.
In this section, we move from the one-dimensional syntax of arrows in free categories to the two-dimensional syntax of \emph{string diagrams}, the arrows of free \emph{monoidal categories}.
