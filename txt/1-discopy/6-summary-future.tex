%!TEX root = ../../THESIS.tex

\section{Summary \& future work} \label{section:summary-and-future}

This chapter gave a comprehensive overview of DisCoPy and the mathematics behind its design principles: we take the definitions of category theory (as strictly and freely as possible) and translate them into a Pythonic syntax.
Figure~\ref{fig:summary} summarises the different modules and their inheritance hierarchy, implementing a subset of the hierarchy of graphical languages surveyed by Selinger~\cite{Selinger10}.
We hope it may be useful both as an introduction to monoidal categories for the Python programmer, and an introduction to Python programming for the applied category theorist.

\begin{figure}[H]% https://q.uiver.app/?q=WzAsMTcsWzEsMCwiXFxweXtjYXR9XjEiXSxbMSwxLCJcXHB5e21vbm9pZGFsfV4yIl0sWzAsMiwiXFxweXtjbG9zZWR9Il0sWzIsMiwiXFxweXtzcGFjaWFsfV4zIl0sWzIsMywiXFxweXticmFpZGVkfSJdLFsyLDQsIlxccHl7c3ltbWV0cmljfV4zIl0sWzAsMywiXFxweXtyaWdpZH1eMiJdLFswLDQsIlxccHl7cGl2b3RhbH0iXSxbMCw1LCJcXHB5e3RvcnRpbGV9Il0sWzAsNiwiXFxweXtjb21wYWN0fV4zIl0sWzAsNywiXFxweXtoeXBlcmdyYXBofV4zIl0sWzIsNSwiXFxweXtjYXJ0ZXNpYW59Il0sWzEsNiwiXFxweXt0cmFjZWR9XjMiXSxbMSw4LCJcXHB5e2JpcHJvZHVjdHN9XjEiXSxbMCw4LCJcXHB5e3RlbnNvcn0iXSxbMSw3LCJcXHB5e21hdHJpeH0iXSxbMiw3LCJcXHB5e3B5dGhvbn0iXSxbMCwxXSxbMSwyXSxbMSwzXSxbMiw2XSxbMyw0XSxbNiw3XSxbMTUsMTRdLFsxNSwxM10sWzcsOF0sWzgsOV0sWzksMTBdLFsxMCwxNF0sWzQsNV0sWzUsMTFdLFs1LDEyXSxbMTIsOV0sWzExLDE2XSxbMTEsMTVdLFsxMiwxNV0sWzIsMTZdXQ==
\centering
\begin{tikzcd}
	& {\py{cat}^{\ref{section:cat}}} \\
	& {\py{monoidal}^{\ref{section:monoidal}}} \\
	{\py{closed}^{\ref{subsection:closed}}} && {\py{spacial}^{\ref{subsection:hypergraph-vs-premonoidal}}} \\
	{\py{rigid}^{\ref{subsection:rigid}}} && {\py{braided}^{\ref{subsection:symmetric}}} \\
	{\py{pivotal}^{\ref{subsection:rigid}}} && {\py{symmetric}^{\ref{subsection:symmetric}, \ref{subsection:hypergraph-vs-premonoidal}}} \\
	{\py{tortile}^{\ref{subsection:symmetric}}} && {\py{cartesian}^{\ref{subsection:cartesian}}} \\
	{\py{compact}^{\ref{subsection:symmetric}, \ref{subsection:hypergraph-vs-premonoidal}}} & {\py{traced}^{\ref{subsection:hypergraph-vs-premonoidal}}} \\
	{\py{hypergraph}^{\ref{subsection:hypergraph}, \ref{subsection:hypergraph-vs-premonoidal}}} & {\py{matrix}} & {\py{python}} \\
	{\py{tensor}} & {\py{biproducts}^{\ref{subsection:biproducts}}}
	\arrow[from=1-2, to=2-2]
	\arrow[from=2-2, to=3-1]
	\arrow[from=2-2, to=3-3]
	\arrow[from=3-1, to=4-1]
	\arrow[from=3-3, to=4-3]
	\arrow[from=4-1, to=5-1]
	\arrow[from=8-2, to=9-1]
	\arrow[from=8-2, to=9-2]
	\arrow[from=5-1, to=6-1]
	\arrow[from=6-1, to=7-1]
	\arrow[from=7-1, to=8-1]
	\arrow[from=8-1, to=9-1]
	\arrow[from=4-3, to=5-3]
	\arrow[from=5-3, to=6-3]
	\arrow[from=5-3, to=7-2]
	\arrow[from=7-2, to=7-1]
	\arrow[from=6-3, to=8-3]
	\arrow[from=6-3, to=8-2]
	\arrow[from=7-2, to=8-2]
    \arrow[from=7-2, to=8-3]
	\arrow[from=3-1, to=8-3]
    \arrow[from=4-3, to=6-1]
\end{tikzcd}
\caption{DisCoPy's modules and the sections where they are discussed.}
\label{fig:summary}
\end{figure}

We list but a few of the many potential directions for further developments.

\begin{itemize}
\item DisCoPy was implemented mainly with correctness in mind, thus there is much room for improving performance.
For now, this has not been quite necessary since the diagrams we manipulate are exponentially smaller than the computation they represent.
However if we want to implement any serious rewriting efficiently, we will need to port the core algorithms to a lower-level language such as Rust~\cite{KlabnikNichols19} and wrap them with Python bindings.
This strategy has improved the time performance of PyZX by over 4000 on a small benchmark consisting of the fusion of 1 million spiders\footnote{\url{https://github.com/quantomatic/quizx}}.

\item As we mentioned in section~\ref{subsection:tacit-to-explicit}, DisCoPy uses a \emph{point-free}, \emph{tacit programming} style which can get very verbose as soon as diagrams have more than a few boxes.
One of the features in our backlog is implementing an \emph{explicit} syntax where diagrams are defined as decorated Python functions taking the wires in their domain as argument, applying boxes to them and returning their codomain.
We already have a working version of this for planar diagrams, it would be straightforward to extend it to any cartesian diagram where we can swap, copy and discard arguments.
What would be less straightforward is to extend it to the syntax of structures beyond cartesian: cocartesian (control flow), closed (higher-order functions) and traced (iteration and recursion).
One starting point for this, rather than reinventing the wheel, would be to use JAX~\cite{BradburyEtAl20} expressions as an intermediate language between pure Python and diagrams.

\item There are many more ways we can interpret diagrams as code, i.e. many more functors into concrete categories we can implement.
One example is probabilistic functions which can be modeled as arrows of \emph{Markov categories}~\cite{FritzEtAl20a} where the objects have comonoids but only the counit is natural.
DisCoPy has already been interfaced with the probabilistic programming language Pyro~\cite{BinghamEtAl19} in order to learn both the structure and the parameters of a machine learning model end-to-end~\cite{Sennesh20}.

\item Some of these concrete categories will not be strict: $(x \otimes y) \otimes z$ and $x \otimes (y \otimes z)$ can represent two different ways of storing the same data, and using one versus the other may have an impact on performance.
Diagrams for non-strict monoidal categories have been used to give an elementary proof of MacLane's coherence theorem for monoidal categories~\cite{WilsonEtAl22}.
We have also drawn them throughout this thesis when discussing coherence for rigid, braided and hypergraph categories.
For now we had to cheat and manually define a new type \py{xy} with boxes from \py{x @ y} to \py{xy} an back, better support for such monoidal coherence is also in the backlog.

\item There are many more constructions from category theory that could be implemented in DisCoPy.
One example is the \emph{$\mathbf{Int}$ construction} which defines the free compact-closed category generated by a traced symmetric category $C$~\cite[Section~4]{JoyalEtAl96}.
Generalising the way the integers $\Z$ are constructed as a quotient of pairs of natural numbers, the objects of $\mathbf{Int}(C)$ are given by pairs of objects in $C$, the arrows by pairs of arrows going in opposite direction and their composition by the trace.
The $\mathbf{Int}$ construction allows to reason about \emph{bidirectional processes} such as \emph{optics} in functional programming~\cite{LavoreRoman19}.
It is also related to the notion of \emph{combs} or \emph{open diagrams}~\cite{Roman20a} which have been used to reason about processes with feedback~\cite{Roman20} as well as causal quantum processes~\cite{KissingerUijlen19}.
Other examples include \emph{open learners}~\cite{FongJohnson19} and \emph{open games}~\cite{Hedges17,Hedges19a} which formalise machine learning and game theory in terms of monoidal categories with some notion of bidirectionality.
\end{itemize}
