%!TEX root = ../../THESIS.tex

\section{Summary \& future work} \label{section:summary-and-future}

This chapter gave a comprehensive overview of DisCoPy and the mathematics behind its design principles: we take the definitions of category theory (as strictly and freely as possible) and translate them into a Pythonic syntax.
Figure~\ref{fig:summary} summarises the different modules and their inheritance hierarchy, implementing a subset of the hierarchy of graphical languages surveyed by Selinger~\cite{Selinger10}.
We hope it may be useful both as an introduction to monoidal categories for the Python programmer, and an introduction to Python programming for the applied category theorist.

\begin{figure}[H]% https://q.uiver.app/?q=WzAsMTcsWzEsMCwiXFxweXtjYXR9XjEiXSxbMSwxLCJcXHB5e21vbm9pZGFsfV4yIl0sWzAsMiwiXFxweXtjbG9zZWR9Il0sWzIsMiwiXFxweXtzcGFjaWFsfV4zIl0sWzIsMywiXFxweXticmFpZGVkfSJdLFsyLDQsIlxccHl7c3ltbWV0cmljfV4zIl0sWzAsMywiXFxweXtyaWdpZH1eMiJdLFswLDQsIlxccHl7cGl2b3RhbH0iXSxbMCw1LCJcXHB5e3RvcnRpbGV9Il0sWzAsNiwiXFxweXtjb21wYWN0fV4zIl0sWzAsNywiXFxweXtoeXBlcmdyYXBofV4zIl0sWzIsNSwiXFxweXtjYXJ0ZXNpYW59Il0sWzEsNiwiXFxweXt0cmFjZWR9XjMiXSxbMSw4LCJcXHB5e2JpcHJvZHVjdHN9XjEiXSxbMCw4LCJcXHB5e3RlbnNvcn0iXSxbMSw3LCJcXHB5e21hdHJpeH0iXSxbMiw3LCJcXHB5e3B5dGhvbn0iXSxbMCwxXSxbMSwyXSxbMSwzXSxbMiw2XSxbMyw0XSxbNiw3XSxbMTUsMTRdLFsxNSwxM10sWzcsOF0sWzgsOV0sWzksMTBdLFsxMCwxNF0sWzQsNV0sWzUsMTFdLFs1LDEyXSxbMTIsOV0sWzExLDE2XSxbMTEsMTVdLFsxMiwxNV0sWzIsMTZdXQ==
\begin{tikzcd}
	& {\py{cat}^{\ref{section:cat}}} \\
	& {\py{monoidal}^{\ref{section:monoidal}}} \\
	{\py{closed}^{\ref{subsection:closed}}} && {\py{spacial}^{\ref{subsection:hypergraph-vs-premonoidal}}} \\
	{\py{rigid}^{\ref{subsection:rigid}}} && {\py{braided}^{\ref{subsection:symmetric}}} \\
	{\py{pivotal}^{\ref{subsection:rigid}}} && {\py{symmetric}^{\ref{subsection:symmetric}, \ref{subsection:hypergraph-diagrams}}} \\
	{\py{tortile}^{\ref{subsection:symmetric}}} && {\py{cartesian}^{\ref{subsection:cartesian}}} \\
	{\py{compact}^{\ref{subsection:symmetric}, \ref{subsection:hypergraph-diagrams}}} & {\py{traced}^{\ref{subsection:hypergraph-diagrams}}} \\
	{\py{hypergraph}^{\ref{subsection:hypergraph}, \ref{subsection:hypergraph-diagrams}}} & {\py{matrix}} & {\py{python}} \\
	{\py{tensor}} & {\py{biproducts}^{\ref{subsection:biproducts}}}
	\arrow[from=1-2, to=2-2]
	\arrow[from=2-2, to=3-1]
	\arrow[from=2-2, to=3-3]
	\arrow[from=3-1, to=4-1]
	\arrow[from=3-3, to=4-3]
	\arrow[from=4-1, to=5-1]
	\arrow[from=8-2, to=9-1]
	\arrow[from=8-2, to=9-2]
	\arrow[from=5-1, to=6-1]
	\arrow[from=6-1, to=7-1]
	\arrow[from=7-1, to=8-1]
	\arrow[from=8-1, to=9-1]
	\arrow[from=4-3, to=5-3]
	\arrow[from=5-3, to=6-3]
	\arrow[from=5-3, to=7-2]
	\arrow[from=7-2, to=7-1]
	\arrow[from=6-3, to=8-3]
	\arrow[from=6-3, to=8-2]
	\arrow[from=7-2, to=8-2]
	\arrow[from=3-1, to=8-3]
    \arrow[from=4-3, to=6-1]
\end{tikzcd}
\caption{DisCoPy's modules and the sections where they are discussed.}
\label{fig:summary}
\end{figure}

We list but a few of the many potential directions for further developments.

\begin{itemize}
\item DisCoPy was implemented only with correctness in mind, thus there is much room for improving performance.
For now, this has not been quite necessary since the diagrams we manipulate are exponentially smaller than the computation they represent.
If we want to implement any serious rewriting efficiently, we will need to port the core algorithms to a lower-level language such as Rust~\cite{KlabnikNichols19} and wrap them with Python bindings.
This strategy has improved the time performance of PyZX by over four thousand on a small benchmark consisting of the fusion of one million spiders\footnote{\url{https://github.com/quantomatic/quizx}}.
%
% \item Markov categories, probabilistic computation, causality.
%
% \item Teleological categories, open games, learners and optics
\end{itemize}
