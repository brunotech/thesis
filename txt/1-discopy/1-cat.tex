%!TEX root = ../../THESIS.tex

\section{Categories in Python}

What are categories and how can they be useful to the Python programmer?
This section will answer this question by taking the standard mathematical definitions and breaking them into \emph{data}, which can be translated into Python code, and \emph{axioms}, which cannot be formally verified in Python, but can be translated into test cases.
The data for a category is given by a tuple $C = (C_0, C_1, \dom, \cod, \id, \then)$ where:
\begin{itemize}
\item $C_0$ and $C_1$ are classes of \emph{objects} and \emph{arrows} respectively,
\item $\dom, \cod : C_1 \to C_0$ are functions called \emph{domain} and \emph{codomain}, we write $f : x \to y$ whenever $\dom(f) = x$ and $\cod(f) = y$ for $f \in C_1$ and $x, y \in C_0$,
\item $\id : C_0 \to C_1$ is a function called \emph{identity},
\item $\then : C_1 \times C_1 \to C_1$ is a partial function called \emph{composition}, denoted by $(\fcmp)$.
\end{itemize}
such that the following axioms hold:
\begin{itemize}
\item $\id(x) : x \to x$ for all objects $x \in C_0$,
\item for all arrows $f, g \in C_1$, the composition $f \fcmp g$ is defined iff $\cod(f) = \dom(g)$, moreover we have $f \fcmp g : \dom(f) \to \cod(g)$,
\item $\id(\dom(f)) \fcmp f = f = f \fcmp \id(\cod(f))$ for all arrows $f \in C_1$,
\item $f \fcmp (g \fcmp h) = (f \fcmp g) \fcmp h$ whenever either side is defined for $f, g, h \in C_1$.
\end{itemize}
Note that we play with the overloaded meaning of the word \emph{class}: we use it to mean both a mathematical collection that need not be a set, and a Python class with its methods and attributes.
Reading it in the latter sense, $\dom$ and $\cod$ are \emph{attributes} of the arrow class, $\then$ is a \emph{method}, $\id$ is a \emph{static method}.
Thus, implementing a category in Python means nothing more than subclassing the  abstract classes \py{Ob} and \py{Arrow} of figure~\ref{fig:abstract classes}, and then checking that the axioms hold via some (necessarily non-exhaustive) software tests.
The data for a functor $F : C \to D$ is given by a pair of overloaded functions $F_0 : C_0 \to D_0$ and $F_1 : C_1 \to D_1$ such that:
\begin{itemize}
    \item $F(\dom(f)) = \dom(F(f))$ and $F(\cod(f)) = \cod(F(f))$ for all $f \in C_1$,
    \item $F(\id(x)) = \id(F(x))$ and $F(f \fcmp g) = F(f) \fcmp F(g)$ for all $x \in C_0$ and $f, g \in C_1$.
\end{itemize}
Thus, implementing a functor in Python amounts to subclassing the \py{Functor} class of figure~\ref{fig:abstract classes} (and then checking that the axioms hold).

\begin{figure}[h]
\caption{Abstract Python classes for categories and functors.}
\label{fig:abstract classes}
\begin{python}
class Ob:
    ...

class Arrow:
    dom : Ob
    cod : Ob

    @staticmethod
    def id(x : Ob) -> Arrow:
        ...

    def then(self, other : Arrow) -> Arrow:
        ...

class Functor:
    @overload
    def __call__(self, x : Ob) -> Ob:
        ...

    @overload
    def __call__(self, f : Arrow) -> Arrow:
        ...
\end{python}
\end{figure}

\begin{example}
We can define the category $\mathbf{Python}$ with objects the class of all Python types and arrows the class of all Python functions.
Domain and codomain of may be extracted from type annotations.
Identity may given by \py{lambda *xs: xs} and the composition by \py{lambda f, g: lambda *xs: f(*g(*xs)))}. (The star takes care of functions with multiple arguments.)
However, equality of functions in Python is undecidable so there will be no way to check the axioms hold in general.
We can define endofunctors $\mathbf{Python} \to \mathbf{Python}$ the same way they are defined in Haskell.
For example, we can define a list functor which sends a type \py{t} to \py{List[t]} and a function \py{f} to \py{lambda *xs: map(f, xs)}.
\end{example}

\begin{example}
When the class of objects and arrows are in fact sets, $C$ is called a \emph{small category}.
For example, the category $\mathbf{FinSet}$ has the set of all finite sets as objects and the set of all functions between them as arrows.
This time equality of functions between finite sets is decidable, so we can write unit tests that check that the axioms hold on specific examples.
\end{example}

\begin{example}
When the class of objects and arrow are finite sets, we can draw the category as a finite graph with objects as vertices and arrows as edges, together with a list of equations between the paths.
A functor $F : C \to D$ from such a finite category $C$ is called a \emph{commutative diagram} in $D$.
They play the same role in category theory as equations in set theory.
For example, take the following commutative diagram in $\mathbf{Python}$:
\[ \begin{tikzcd}
\py{int}
\ar{rrrrrr}{\py{lambda x: x * (x - 1) // 2}}
\ar{rrrd}[']{\py{range}}
& & & & & & \py{int} \\
& & & \py{Iterable}
\ar{urrr}[']{\py{sum}} & & &
\end{tikzcd}
\]
\end{example}

\begin{example}
The category $\mathbf{Mat}_\S$ has natural numbers as objects and $n \times m$ matrices with values in $\S$ as arrows $n \to m$.
The identity and composition are given by the identity matrix and matrix multiplication respectively.
In order for matrix multiplication to be well-defined and for $\mathbf{Mat}_\S$ to be a category, the scalars $\S$ should have the structure of a \emph{rig} (a riNg without Negatives).
When the scalars are complex numbers $\S = \C$, the category $\mathbf{Mat}_\C$ is equivalent to the category of finite dimensional vector spaces and linear maps.
When the scalars are Booleans with disjunction and conjunction as addition and multiplication $\S = \B$, the category $\mathbf{Mat}_\B$ is equivalent to the category of finite sets and relations.
There is a faithful functor (i.e. injective on arrows with the same domain and codomain) $\mathbf{FinSet} \to \mathbf{Mat}_\B$ which sends finite sets to their cardinality and functions to their graph.
\end{example}

\begin{example}
The category $\mathbf{Circ}$ has natural numbers as objects and $n$-qubit quantum circuits as arrows $n \to n$.
There is a functor $\mathtt{eval} : \mathbf{Circ} \to \mathbf{Mat}_\C$ which sends $n$ qubits to $2^n$ dimensions and evaluates each circuit to its unitary matrix.
\end{example}

\begin{example}
Just about any class of mathematical structures as objects and their homomorphisms as arrows will yield a category.
For example, the category $\mathbf{Set}$ of sets and functions, the category $\mathbf{Mon}$ of monoids and homomorphisms, the category $\mathbf{Cat}$ of small categories and functors, etc.
\end{example}

% A \emph{quiver}\footnote
% {Quivers are directed multigraphs \emph{with an attitude}~\cite{zotero-3194} they are also synonymous with Selinger's simple signatures~\cite{Selinger10}.}
% $\Sigma$ is a tuple $(\Sigma_0, \Sigma_1, \dom, \cod)$ where:
% \begin{itemize}
%     \item $\Sigma_0$ and $\Sigma_1$ are sets of \emph{objects} and \emph{arrows} respectively,
%     \item $\dom : \Sigma_1 \to \Sigma_0$ is the \emph{domain} function, $\cod : \Sigma_1 \to \Sigma_0$ is the \emph{codomain}.
% \end{itemize}
% We write $f : x \to y$ or $x \xto{f} y$ for $x, y \in \Sigma_0$ and $f \in \Sigma_1$ with $\dom(f) = x$ and $\cod(f) = x$.
% A morphism of quivers $f : \Sigma \to \Sigma'$ is a pair of functions $f_0 : \Sigma_0 \to \Sigma'_0$ and $f_1 : \Sigma_1 \to \Sigma'_1$ with $f_0 \circ \dom = \dom \circ f_1$ and $f_0 \circ \cod = \cod \circ f_1$.
%
% A \emph{category} $C$ is a quiver with
% \begin{itemize}
%     \item a composition $C_1
% \end{itemize}
