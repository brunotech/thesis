%!TEX root = ../../THESIS.tex

\subsection{Products \& coproducts} \label{subsection:cartesian}

With hypergraph diagrams, we have enough syntax to discuss quantum protocols and first-order logic.
However, the spiders of hypergraph categories are of no use if we want to interpret our diagrams as (pure) Python functions with \py{tuple} as tensor.
Indeed, $\mathbf{Pyth}$ has the property that every function $f : x \to y \otimes z$ into a composite system $y \otimes z$ is in fact a tensor product $f = f_0 \otimes f_1$ of two separate functions $f_0 : x \to y$ and $f_1 : x \to z$.
If a Python type $x$ had caps (let alone spiders) then we could break them in two with the consequence that the identity function on $x$ is constant, i.e. $x$ is trivial~\cite[Proposition~4.76]{CoeckeKissinger17}.
Moreover, there is only one (pure) effect of every type, discarding it.
Thus if a Python type $x$ had cups then we could break them apart as well with the same consequence: only the trivial Python type can have spiders.
A similar argument destroys our hopes for time reversal in Python: if we had a monoidal dagger on $\mathbf{Pyth}$, every state would be equal to every other.

Now if we go back to the intuition of diagrams as pipelines and their wires as carrying data, not all might be lost about spiders.
Indeed, it makes sense to split a data-carrying wire: it means we are copying information.
Closing a data-carrying wire is the counit of the copying comonoid, it means we are deleting information.
In this context, the special condition would translate as follows: if we copy some data then merge the two copies back together, then we haven't done anything.
In order for the spider fusion equations to hold, we would need the monoid to take any two inputs and assert that they are equal or abort the computation otherwise, i.e. we would need side effects.
Even more weirdly, we would need the unit of the monoid to be equal to anything else.

Rather than complaining that classical computing is weird because we cannot coherently merge data back together, we should embrace this as a feature, not a bug: in Python we can copy and discard data (at least assuming that we have enough RAM and that the garbage collector is doing its job).
This means we can still keep the comonoid half of our spiders, forget that they are spiders and come to realise that they are in fact \emph{natural comonoids}, i.e. every function is a comonoid homomorphism.
Indeed, the functions \py{copy = lambda *xs: xs + xs} and \py{delete = lambda *xs: ()} define a pair of natural transformations $x \to x \otimes x$ and $x \to 1$ in $\mathbf{Pyth}$:
\begin{itemize}
    \item \py{copy(f(xs)) == f(copy(xs)[:n]), f(copy(xs)[n:])}
    \item \py{delete(f(xs)) == delete(xs)}
\end{itemize}
for all pure functions \py{f} and inputs \py{xs} with \py{n = len(xs)}.
Once drawn as a diagram, the naturality equations for comonoids allows us to either copy or delete boxes by passing them through either the coproduct or the counit.

\ctikzfig{img/cartesian/naturality}

A \emph{cartesian category} is a symmetric category with coherent, natural commutative comonoids.
The category $\mathbf{Pyth}$ is an example of cartesian category, as well as the categories $\mathbf{Set}$, $\mathbf{Mon}$, $\mathbf{Cat}$, $\mathbf{MonCat}$, etc.
The category $\mathbf{Mat}_\S$ is also a cartesian category with the direct sum as tensor.
Our definition of cartesian is convenient if we want to draw string diagrams and interpret them as functions but it is rather cumbersome: checking that a given category fits the definition involves a lot of structure (tensor, swaps and comonoids) and many axioms relating them.
In practice, we usually take an equivalent definition: a category $C$ is cartesian if it has \emph{categorical products} and a \emph{terminal object}.
An object $1 \in C_0$ is terminal if there is a unique arrow $\counit(x) : x \to 1$ from each object $C_0$.
An object $x_0 \times x_1 \in C_0$ is the product of two objects $x_0, x_1 \in C_0$ if it comes equipped with a pair of arrows $\pi_0 : x_0 \times x_1 \to x_0$ and $\pi_1 : x_0 \times x_1 \to x_1$
such that for all pairs of arrows $f_0 : y \to x_0$ and $f_1 : y \to x_1$ there is a unique $f = \langle f_0, f_1 \rangle : y \to x_0 \times x_1$ such that $f \fcmp \pi_0 = f_0$ and $f \fcmp \pi_1 = f_1$.
These definitions are usually drawn as commutative diagrams where the full lines are universally quantified and the dotted line is uniquely existentially quantified.

From these two \emph{universal properties} we can deduce that terminal objects and categorical products are unique up to a unique isomorphism and that they form a monoid on objects $(C_0, \times, 1)$, up to natural isomorphism.
Given two arrows $f : a \to b$ and $g : c \to d$ we have two arrows $\pi_0 \fcmp f : a \times c \to b$ and $\pi_1 \fcmp g : a \times c \to d$, thus there is a unique $f \times  g = \langle \pi_0 \fcmp f, \pi_1 \fcmp g \rangle : a \times b \to c \times d$.
One can show that this makes the category $C$ a monoidal category, i.e. $(C_1, \times, \id(1))$ is a monoid and the interchange law holds.
Furthermore, we can show $C$ is symmetric with the swaps given by $S(x, y) = \langle \pi_1, \pi_0 \rangle : x \times y \to y \times x$.
Finally, we can show $C$ has coherent natural commutative comonoids given by $\ttsplit(x) = \langle \id(x), \id(x) \rangle : x \to x \times x$ and $\counit(x) : x \to 1$.

In the other direction, if $C$ has coherent natural commutative comonoids we can deduce that $1$ a terminal object from the naturality of the counit.
For any arrows $f_0 : y \to x_0$ and $f_1 : y \to x_1$ we can define
\begin{itemize}
\item $\langle f_0, f_1 \rangle = \ttsplit(y) \fcmp f_0 \otimes f_1$,
\item $\pi_0 = \id(x_0) \otimes \counit(x_1)$ and $\pi_1 = \counit(x_0) \otimes \id(x_1)$,
\end{itemize}
and show that $\otimes = \times$ is in fact a categorical product, see Selinger's survey~\cite[Section 6.1]{Selinger10}.
A functor is cartesian when it preserves the categorical product, or equivalently if it is a symmetric functor that preserves the comonoid.
This defines a category $\mathbf{CCat}$ of cartesian categories and functors.
From lemma~\ref{lemma:foo-monoidal} we can assume that cartesian categories are free-on-objects, i.e. the monoid axioms for objects are equalities rather than natural transformations.
Thus, we get a forgetful functor $U : \mathbf{CCat} \to \mathbf{MonSig}$ with its left-adjoint given by a quotient of the free symmetric category $F^C(\Sigma) = F^S(\Sigma \cup \ttsplit \cup \counit) / R$ with the relations $R$ given by the naturality equations for each box.

Taking the opposite definition, a \emph{cocartesian category} is one with a categorical coproduct, or equivalently with a coherent natural commutative monoid.
For example, the category $\mathbf{Set}$ is cocartesian with the disjoint union, the category $\mathbf{Pyth}$ is cocartesian with the tagged union: the merging function takes a tagged element of an n-fold disjoint union and forgets the tag.
While cartesian structures can be thought of in terms of product types and data copying, cocartesian structures formalise tagged unions and conditional branching.
Indeed, when we interpret cocartesian diagrams in $\mathbf{Pyth}$ parallel wires encode the different branches of a program, merging two wires of the same type means forgetting the difference between two branches.

DisCoPy implements free (co)cartesian categories with a subclass of \py{Box} for making (merging) \py{n} copies of a type \py{x} of length one.
The class method \py{Diagram.copy} allows to make \py{n} copies of an arbitrary type \py{x} by calling the \py{coherence} subroutine of the previous section, it takes an optional argument \py{is_dagger} that allows to merge copies instead.
Cartesian functors take \py{Copy} boxes of its domain to the \py{copy} method of its codomain.

\begin{python}
{\normalfont Implementation of free (co)cartesian categories and functors.}

\begin{minted}{python}
class Copyable:
    @classmethod
    def discard(cls, x: Ty, is_dagger=False) -> Copyable:
        return cls.copy(x, 0, is_dagger)

    @classmethod
    def merge(cls, x: Ty, n=2, is_dagger=False) -> Copyable:
        return cls.copy(x, n, not is_dagger)

    @classmethod
    def unit(cls, x: Ty): return cls.discard(x, is_dagger=True)

class Diagram(Copyable, symmetric.Diagram):
    @classmethod
    def copy(cls, x: Ty, n=2, is_dagger=False) -> Diagram:
        def factory(a, b, y):
            assert b == 1 if is_dagger else a == 1
            return Copy(a, y)[::-1] if is_dagger else Copy(b, x)
        a, b = (n, 1) if is_dagger else (1, n)
        return coherence(factory)(cls, a, b, x)

class Box(symmetric.Box, Diagram):
    cast = Diagram.cast

class Copy(Box):
    def __init__(self, x: Ty, n: int = 2, is_dagger=False):
        assert len(x) == 1
        name = "Copy({}, {}){}".format(x, n, "[::-1]" if is_dagger else "")
        dom, cod = (x ** n, x) if is_dagger else (x, x ** n)
        super().__init__(name, dom, cod, is_dagger)

    def dagger(self):
        x = self.cod if self.is_dagger else self.dom
        n = len(self.dom) if self.is_dagger else len(self.cod)
        return Copy(x, n, not self.is_dagger)

class Functor(symmetric.Functor):
    dom = cod = Category(monoidal.Ty, Diagram)

    def __call__(self, other):
        if isinstance(other, Copy):
            factory = getattr(self.cod.ar, "merge" if other.is_dagger else "copy")
            return factory(self(other.dom), len(other.cod))
        return super().__call__(other)
\end{minted}
\end{python}

\begin{example}
In a cartesian category, every monoid is automatically a \emph{bialgebra}, i.e. the monoid is a comonoid homomorphism or equivalently, the comonoid is a homomorphism for the monoid.
When furthermore the monoid has an inverse, then it is automatically a \emph{Hopf algebra}, the generalisation of groups to arbitrary monoidal categories.

\begin{minted}{python}
x = Ty('x')
add, minus, zero = Box('+', x @ x, x), Box('-', x, x), Box('0', Ty(), x)
copy, discard = Diagram.copy(x), Diagram.discard(x)

drawing.equation(add >> copy, copy @ copy >> x @ Braid(x, x) @ x >> add @ add)
drawing.equation(zero >> copy, zero @ zero)
\end{minted}
\begin{center}
\tikzfig{img/cartesian/add-hom-coproduct}
\hfill
\tikzfig{img/cartesian/zero-hom-coproduct}
\end{center}
\begin{minted}{python}
drawing.equation(add >> discard, discard @ discard)
drawing.equation(zero >> discard, Diagram.id(Ty()))
\end{minted}
\begin{center}
\tikzfig{img/cartesian/add-hom-counit}
\hfill
\tikzfig{img/cartesian/zero-hom-counit}
\end{center}
\begin{minted}{python}
drawing.equation(copy >> minus @ x >> add, discard >> zero, copy >> x @ minus >> add)
\end{minted}
\ctikzfig{img/cartesian/hopf}

A bialgebra which is also a Frobenius algebra is necessarily trivial, i.e. isomorphic to the unit.

\begin{minted}{python}
drawing.equation(
    Diagram.id(x),
    x @ zero >> x @ copy >> add @ x >> discard @ x,
    x @ zero @ zero >> discard @ discard @ x,
    discard >> zero)
\end{minted}

\ctikzfig{img/cartesian/bialgebra-frobenius-implies-trivial}
\end{example}

A cartesian category that is also a PROP is called a \emph{Lawvere theory}~\cite{Lawvere63}, they were first introduced as a high-level language for \emph{universal algebra}. take the boxes $x^n \to x$ to be the primitive $n$-ary operations of your language (e.g. for rigs we have unary boxes for $0$ and $1$, binary boxes for $+$ and $\times$) then the diagrams in the free Lawvere theory are all the terms of your language.
We can write down any universally quantified axiom as a relation between diagrams and the cartesian functors from the resulting quotient to $\mathbf{Set}$ are \emph{algebras}, i.e. sets equipped with operations that satisfy the axioms.
The natural transformations between models are precisely the homomorphisms between the algebras, i.e. the functions that commute with the operations.
If we add colours back in and allow many generating objects, we can define many-sorted theories such as that of modules, with a ring acting on a group.
If we take every pair of objects $(x, y) \in C_0 \times C_0$ as colour and a box $(x, y) \otimes (y, z) \to (x, z)$ for every possible composition, then we can even define the Lawvere theory of categories with $C_0$ as objects.
Thus, categories (with some fixed objects) can also be seen as the functors from this Lawvere theory to $\mathbf{Set}$, functors can also be seen as the natural transformations between such functors.

The free Lawvere theory with no boxes (only swaps, coproducts and counits) is equivalent to $\mathbf{FinSet}^{op}$, the opposite category to finite sets and functions, with the disjoint union as tensor.
Indeed, a cartesian diagram $f : x^n \to x^m$ can be seen as the graph of a function from the $m$ to $n$ elements.
This free Lawvere theory is also called the \emph{theory of equality}, a functor from it to $\mathbf{Set}$ (i.e. an algebra for the theory) is just a set with its equality relation, a natural transformation between those functors is just a function.
Thus, equality of cartesian diagrams with no boxes reduces to equality of functions between finite sets, which can be implemented as equality of finite dictionaries in Python.
It is easy to show that the rules for naturality are confluent and terminating when applied from left to right, i.e. we copy (delete) every box by passing it down through all possible coproducts (counits).
At each rewrite step there is one fewer box node above a comonoid node, thus we can reduce the word problem for free cartesian categories to that of free symmetric categories and then to graph isomorphism.

\begin{python}
Implementation of the free cocartesian category $\mathbf{FinSet}$ with \py{int} as objects and \py{Dict} as arrows.

\begin{minted}{python}
@dataclass
class Dict(Composable, Tensorable):
    inside: dict[int, int]
    dom: int
    cod: int

    __getitem__ = lambda self, key: self.inside[key]

    def __init__(self, inside, dom, cod):
        assert set(inside.keys()) == set(range(dom))
        assert set(inside.values()) <= set(range(cod))
        self.inside, self.dom, self.cod = inside, dom, cod

    @staticmethod
    def id(x: int = 0): return Dict({i: i for i in range(x)}, x, x)

    @inductive
    def then(self, other: Dict) -> Dict:
        return Dict({i: other[self[i]] for i in range(self.dom)}, self.dom, other.cod)

    @inductive
    def tensor(self, other: Dict) -> Dict:
        inside = {i: self[i] for i in range(self.dom)}
        inside.update({len(self.dom) + i: len(self.cod) + other[i]
                       for i in range(other.dom)})
        return Dict(inside, self.dom + other.dom, self.cod + other.cod)

    @staticmethod
    def merge(x: int, n: int) -> Dict:
        return Dict({i: i % x for i in range(n * x)}, n * x, x)
\end{minted}
\end{python}

\begin{example}
We can check equality of cartesian diagrams with one generating object and no boxes.

\begin{minted}{python}
x = Ty('x')
copy, discard, swap = Diagram.copy(x), Diagram.discard(x), Diagram.swap(x, x)
F = lambda f: Functor({x: 1}, {}, cod=Category(int, Dict))(f.dagger())

assert F(copy >> discard @ x) == F(Diagram.id(x)) == F(copy >> x @ discard)
assert F(copy >> copy @ x) == F(Diagram.copy(x, 3)) == F(copy >> x @ copy)
assert F(copy >> swap) == F(copy)
\end{minted}
\end{example}

A category with distinct cartesian and cocartesian structures is an example of a \emph{rig category}, i.e. it has two monoidal structures $\oplus$ and $\otimes$ that satisfy the equations of a rig up to natural isomorphism.
This is the case for the category $\mathbf{Set}$ as well as for $\mathbf{Pyth}$.
The Kronecker product is not a cartesian product for $\mathbf{Mat}_\S$ (since this role is taken by direct sums) but it does form a rig category with the direct sum.
The arrows of free rig categories can be described as (equivalence classes of) three-dimensional \emph{sheet diagrams} where composition, additive and multiplicative tensor are encoded in three orthogonal axes~\cite{ComfortEtAl20}.
These 3d diagrams are a complete language for \emph{dataflow programming}~\cite{Delpeuch20a}, they are not implemented in DisCoPy yet.

The implementation of $\mathbf{Pyth}$ as a cartesian category with \py{tuple} and a cocartesian category with \py{tagged_union} is straightforward.
What is more challenging is to make both monoidal structures fit in the same class, i.e. implementing a rig category.
Indeed, if we take $(\mathbf{Pyth}, \py{tuple})$ as the foo-monoidal category implemented in section~\ref{section:monoidal}, \py{dom: list[type]} and \py{cod: list[type]} are interpreted as a \py{tuple}.
On the other hand, if we implement $(\mathbf{Pyth}, \py{tagged_union})$ as a foo-monoidal category, lists of types will be interpreted as a \py{TaggedUnion} instead.
Hence, we need to add an optional argument \py{is_additive: bool} which tells us how we should interpret \py{dom} and \py{cod}, and a pair of methods for converting additive into multiplicative functions.
In order to keep the syntax simple, we keep the notation \py{f @ g} for \py{tuple} and override the vertical line symbol \py{f | g} to denote the \py{tagged_union}.
We use the same syntax for the direct sum of two tensors.

\begin{python}\label{listing:python-co-cartesian}
{\normalfont Implementation of $\mathbf{Pyth}$ as a cartesian category with \py{tuple} and a cocartesian category with \py{tagged_union}.}

\begin{minted}{python}
is_tuple = lambda typ: hasattr(typ, "__origin__") and typ.__origin__ is tuple
is_union = lambda typ: hasattr(typ, "__origin__") and typ.__origin__ is TaggedUnion

class Function(monoidal.Function):
    is_additive = False

    @classmethod
    @property
    def additive(cls):
        class C(cls): pass
        C.is_additive = True; return C

    def __repr__(self): super().__repr__().replace(
        "Function", "Function.additive" if self.is_additive else "Function")

    def make_additive(self) -> Function:
        if self.is_additive: return self
        dom, cod = (
            [tuple[types]] if len(types) != 1
            else list(types[0].__args__) if is_union(types[0])
            else types for types in (self.dom, self.cod))
        return Function.additive(self.inside, dom, cod)

    def make_multiplicative(self) -> Function:
        if not self.is_additive: return self
        dom, cod = (
            [TaggedUnion[types]] if len(types) != 1
            else list(types[0].__args__) if is_tuple(types[0])
            else types for types in (self.dom, self.cod))
        return Function(self.inside, dom, cod)

    def tuple(self, other: Function) -> Function:
        if self.is_additive: return self.make_multiplicative().tuple(other)
        if other.is_additive: return self.tuple(other.make_multiplicative())
        return super().tuple(other)

    def tagged_union(self, other: Function) -> Function:
        if not self.is_additive: return self.make_additive().tagged_union(other)
        if not other.is_additive: return self.tagged_union(other.make_additive())
        inside = self.inside if len(self.dom) == 1 and len(other.dom) == 0\
            else other.inside if len(self.dom) == 0 and len(other.dom) == 1\
            else lambda i, x: self(i, x) if i < len(self.dom) else other(i - len(self.dom), x)
        dom, cod = self.dom + other.dom, self.cod + other.cod
        return Function(inside, dom, cod)

    @staticmethod
    def copy(x: list[type], n: int, is_dagger=False):
        if is_dagger: return Function.merge(x, n)
        if n == 1: return Function.id(x)
        return Function(lambda *xs: n * xs, dom=x, cod=x ** n)

    @staticmethod
    def merge(x: list[type], n: int, is_dagger=False):
        if is_dagger: return Function.copy(x, n)
        if n == 1: return Function.additive.id(x)
        return Function.additive(lambda _, xs: xs, dom=x ** n, cod=x)

    @inductive
    def tensor(self, other: Function) -> Function:
        return self.tagged_union(other) if self.is_additive else self.tuple(other)

    __or__ = tagged_union
\end{minted}
\end{python}

\begin{example}
We can implement the architecture of a neural network as a cartesian diagram and its evaluation as a functor to \py{Function}.

\begin{minted}{python}
add = lambda x, n: Box('$+$', x ** n, x)
ReLU = lambda x: Box('$\sigma$', x, x)
weights = [Box('w{}{}'.format(i, j), x, x) for i in range(2) for j in range(2)]
bias = Box('b', Ty(), x)

network = Diagram.copy(x @ x, 2) >> Diagram.tensor(*weights) @ bias >> add(x, 5) >> ReLU(x)

F = Functor(ob={x: int}, ar={
    add(x, 5): lambda *xs: sum(xs),
    ReLU(x): lambda x: max(0, x),
    bias: lambda: -1, **dict(zip(weights, [0, 1, 2, 3]))},
    cod=Category(list[type], Function))

assert F(network)(42, -43) == max(0, sum([42 * 0, -43 * 1, 42 * 2, -43 * 3, -1]))
\end{minted}

We could implement a neural network that can operate on different kinds of data (e.g. discrete versus continuous) using a cocartesian diagram instead.
\end{example}

\subsection{Biproducts}\label{subsection:biproducts}

A category has \emph{biproducts} if the cartesian and cocartesian structures coincide,
a $\dagger$-category has \emph{$\dagger$-biproducts} when furthermore the monoid is the dagger of the comonoid.
This is the case in the $\dagger$-category $\mathbf{Mat}_\S$ with direct sum $\oplus$ as tensor.

\begin{python}
{\normalfont Implementation of $\dagger$-biproducts for $\mathbf{Mat}_\S$
and $\mathbf{Tensor}_\S$ as a rig category with direct sum and Kronecker product.}

\begin{minted}{python}
@Copyable
class Matrix:
    ...
    @classmethod
    def copy(cls, x: int, n: int, is_dagger=False) -> Matrix:
        dom, cod = (x ** n, x) if is_dagger else (x, x ** n)
        inside = [[i % n == j for j in range(x)] for i in range(x ** n)] if is_dagger\
            else [[i == j % n for j in range(x ** n)] for i in range(x)]]
        return cls(inside, dom, cod)

    @classmethod
    def merge(cls, x: int, n: int, is_dagger=False) -> Matrix:
        return cls.copy(x, n, not is_dagger)

    @classmethod
    def basis(cls, x: int, i: int, is_dagger=False) -> Matrix:
        inside = [i % j == 0 for j in range(x)]
        inside = [[val] for val in inside] if is_dagger else [inside]
        dom, cod = (x, x ** 0) if is_dagger else (x ** 0, x)
        return cls(inside, dom, cod)

class Tensor:
    ...
    @inductive
    def direct_sum(self, other: Tensor) -> Tensor:
        old = self.downgrade().direct_sum(other.downgrade())
        return type(self)(old.inside, [old.dom], [old.cod])

    __or__ = direct_sum
\end{minted}
\end{python}

Biproducts and matrices happen to be intimately related.
Indeed, given any category $C$ with sums we can construct its \emph{free biproduct completion} $\mathbf{Mat}_C$ as the monoidal category with objects given by $C_0^\star$ and arrows $f : x \to y$ given by matrices $f_{ij} : x_i \to y_j$ of arrows in $C_1$.
Composition in $\mathbf{Mat}_C$ is an extension of the usual matrix multiplication with composition as product.
In particular, if $C = \S$ is a rig, i.e. a one-object category with sums, then this definition coincides with the usual one.
Any category with biproducts also has sums $f + g : x \to y$ given by $\ttsplit(x) \fcmp f \oplus g \fcmp \merge(x)$ and a zero morphism $0 = \counit(x) \fcmp \unit(y)$.
One can verify that the free completion is indeed the left-adjoint to the forgetful functor from biproducts to sums, see~\cite[Exercise VIII.2.6]{MacLane71}.
Similarly, if $C$ is a $\dagger$-category then $\mathbf{Mat}_C$ is its free $\dagger$-biproduct completion with the element-wise dagger of the transpose.
If $C$ is also a monoidal category, then $\mathbf{Mat}_C$ is a rig category with the tensor given by an extension of the usual Kronecker product with tensor as product.

DisCoPy implements free $\dagger$-biproduct completion as a subclass of \py{Matrix[Diagram]} with \py{list[Ty]} as objects and addition given by the formal sum of diagrams.
We use Python's duck typing to lift the code for composition, tensor and direct sum from \py{int} to \py{list[Ty]}.
We also use a \py{contextmanager} to temporarily replace the multiplication of two \py{Diagram} entries.
Again, we override equality so that diagrams are equal to the matrix of just themselves.
The implementation of biproduct-preserving functors requires some work: given a matrix of diagrams, we construct its image as the sum of the images of its entries, pre- and post-composed by the basis row and column vectors.
The \py{basis} subroutine constructs basis vectors for an arbitrary class using only its methods \py{id}, \py{zero}, \py{copy} and \py{direct_sum}.

\begin{python}
{\normalfont Implementation of free $\dagger$-biproduct completion.}

\begin{minted}{python}
@dataclass
class FakeInt:
    inside: list[Ty] = [Ty()]

    __int__ = lambda self: len(self.inside)
    __iter__ = property(lambda self: self.inside.__iter__)
    __add__ = lambda self, other: FakeInt(self.inside + other.inside)
    __mul__ = lambda self, other: FakeInt([x0 @ x1 for x0 in self.inside for x1 in other])
    __pow__ = lambda self, n: product(n * [self], unit=FakeInt())

class Biproduct(Matrix):
    dtype = Diagram

    def __init__(self, inside: list[list[Diagram]], dom: list[Ty], cod: list[Ty]):
        self.dom, self.cod, self.inside = list(dom), list(cod), [[
            cls.dtype.id(x) if val is 1 and x == y
            else cls.dtype.zero(x, y) if val is 0
            else val for y, val in zip(cod, row)] for x, row in zip(dom, inside)]

    @contextmanager
    def fake_multiplication(self, method):
        tmp, self.dtype.__mul__ = getattr(self.dtype, "__mul__", None), method
        self.dom, self.cod = map(FakeInt, (self.dom, self.cod))
        yield self
        self.dom, self.cod = map(list, (self.dom, self.cod))
        delattr(self.dtype, "__mul__") if tmp is None else setattr(self.dtype, "__mul__", tmp)

    @classmethod
    def cast(cls, old: Diagram):
        if isinstance(old, cls): return old
        return cls([[old]], [old.dom], [old.cod])

    @inductive
    def then(self, other: Biproduct | Diagram) -> Biproduct:
        with self.fake_multiplication(self.dtype.then) as self:
            return Matrix.then(self, self.cast(other))

    @inductive
    def tensor(self, other: Biproduct | Diagram) -> Biproduct:
        with self.fake_multiplication(self.dtype.tensor) as self:
            return Matrix.Kronecker(self, self.cast(other))

    @classmethod
    def copy(cls, x: list[Ty], n: int, is_dagger=False):
        with cls.id(x).fake_multiplication(cls.then) as biproduct:
            return Matrix.copy(biproduct.dom, n, is_dagger)

    @classmethod
    def discard(cls, x, is_dagger=False): return cls.copy(x, 0, is_dagger)

    dagger = lambda self: self.transpose().map(lambda f: f.dagger())
    __or__ = lambda self, other: self.direct_sum(self.cast(other))
    __eq__ = lambda self, other: Matrix.__eq__(self, self.cast(other))

for method in ("self", "other"):
    setattr(Diagram, method, inductive(
        lambda self, other: getattr(self.biproduct.cast(self), method)(other)
        if isinstance(other, Biproduct) else getattr(monoidal.Diagram, method)(other)))
Diagram.direct_sum = lambda self, *others: self.biproduct.cast(self).direct_sum(*other)
Diagram.__or__ = Diagram.direct_sum
Diagram.__eq__ = lambda self, other:\
    other.inside = [[self]] if isinstance(other, Biproduct)\
    else monoidal.Diagram.__eq__(self, other)

@classmethod
def basis(cls, x: list[Ty], i: int):
    terms = [cls.id(x[i]) if i == j else cls.zero(x[i], x[j]) for j in range(len(x))]
    return cls.copy(x[i], len(x)) >> cls.direct_sum(*terms)

class Functor(monoidal.Functor):
    dom = cod = Category(list[Ty], Biproduct)

    def __call__(self, other):
        if isinstance(other, Biproduct):
            result = self(other.zero(other.dom, other.cod))
            for i, row in enumerate(other.inside):
                effect = basis(self.cod.ar, list(map(self, other.dom)), i)[::-1]
                for j, diagram in enumerate(row):
                    state = basis(self.cod.ar, list(map(self, other.cod)), j)
                    result += effect >> self(diagram) >> state
            return result
        return super().__call__(other)
\end{minted}
\end{python}

\begin{example}
We can define the object \py{bit} as the list of two empty types, with \py{true} and \py{false} the two basis states
then we can implement conditional expressions as biproducts.

\begin{minted}{python}
bit = 2 * [Ty()]
true = Biproduct.copy(Ty(), 2) >> Diagram.id(Ty()) | Diagram.zero(Ty(), Ty())
false = Biproduct.copy(Ty(), 2) >> Diagram.zero(Ty(), Ty()) | Diagram.id(Ty())

x, y = Ty('x'), Ty('y')
f, g = Box('f', x, y), Box('g', x, y)
conditional = f | g >> Biproduct.merge(y, 2)

assert true @ x >> conditional == f and false @ x >> conditional == g
\end{minted}
\end{example}

\begin{example}\label{example:biproduct-measurement}
We can implement classical control as a biproduct of two quantum states and measurement as a biproduct of two quantum effects.
When we compose classical control with measurement, we get a matrix where the entries are scalar diagrams.
The squared amplitude of the evaluation of these scalars give us the measurement probabilities for each classical choice of state.

\begin{minted}{python}
control = Biproduct([[Ket(0)], [Ket(1)]], bit, [qubit])
measure = Biproduct([[Bra(0), Bra(1)]], [qubit], bit)

F = Functor(
    ob={qubit: 2}, ar=lambda f: f.eval(),
    cod=Category(list[int], Tensor[complex]))

assert F(control >> H >> measure).inside\
    == [[(Ket(i) >> H >> Bra(j)).eval() for j in [0, 1]] for i in [0, 1]]
\end{minted}
\end{example}

As we mentioned at the end of section~\ref{section:monoidal}, DisCoPy uses a \emph{point-free} syntax and it can be rather tedious to define any complex diagram in this way.
It is straightforward to extend the \py{diagramize} method to cartesian diagrams, so that they can be defined using the standard syntax for Python functions, where we can use arguments any number of times in any order.
Extending it to cocartesian diagrams so that they can be defined using the standard Python syntax for conditionals will likely be more challenging.
Given enough engineering, it would be possible to turn any pure Python function into a diagram, however this will require more structure than just (co)cartesian categories.
Functions with side effects can be seen as arrows in \emph{premonoidal categories} which are the topic of section~\ref{section:premonoidal}, while recursive functions are arrows in \emph{traced categories}, both will be discussed in section~\ref{section:premonoidal}.
Higher-order functions are modeled as arrows in \emph{closed categories}, the topic of the next section.
