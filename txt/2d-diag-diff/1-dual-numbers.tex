%!TEX root = ../../THESIS.tex

\subsection{Dual numbers}\label{1-dual-numbers}

Dual numbers were first introduced by Clifford in 1873 \cite{Clifford73}.
Given a commutative rig (i.e. a riNg without Negatives) $\S$, the rig of dual
numbers $\D[\S]$ extends $\S$ by adjoining a new element $\epsilon$ such that $\epsilon^2 = 0$.
Concretely, elements of $\D[\S]$ are formal sums $s + s' \epsilon$ where
$s$ and $s'$ are scalars in $\S$.
We write $\pi_0, \pi_1 : \D[\S] \to \S$
for the projection on the real and epsilon component respectively.
Addition and multiplication of dual numbers are given by:
\begin{align} \begin{split}\label{linearity}
(a + a' \ \epsilon ) + (b + b' \ \epsilon)
\quad &= \quad (a + b) \s + \s (a + b') \ \epsilon
\end{split}\\
\begin{split}\label{product-rule}
(a + a' \ \epsilon ) \times (b + b' \ \epsilon)
\quad &= \quad (a \times b) \s + \s (a \times b' \ + \ a' \times b) \ \epsilon
\end{split}
\end{align}

A related notion is that of differential rig: a rig $\S$ equipped with a
derivation, i.e. a map $\partial : \S \to \S$ which preserves sums and satisfies
the Leibniz product rule
$\partial(f \times g) = f \times \partial(g) + \partial(f) \times g$ for all
$f, g \in \S$.
An equivalent condition is that the map $f \mapsto f + (\partial f) \epsilon$
is a homomorphism of rigs $\S \to \D[\S]$. The correspondance also works the
other way around: given a homorphism $\partial : \S \to \D[\S]$ such that
$\pi_0 \circ \partial = \id_\S$, projecting on the epsilon component is a
derivation $\pi_1 \circ \partial : \S \to \S$. The motivating example is the rig
of smooth functions $\S = \R \to \R$, where differentiation is a derivation.
Concretely, we can extend any smooth function $f : \R \to \R$ to a
function $f : \D[\R] \to \D[\R]$ over the dual numbers defined by:
\begin{equation}\label{dual-numbers-eq}
f(a + a' \epsilon) \quad = \quad f(a) \s + \s a' \times (\partial f)(a) \epsilon
\end{equation}

We can use equations~\ref{linearity}, \ref{product-rule} and \ref{dual-numbers-eq}
to derive the usual rules for gradients in terms of dual numbers.
For the identity function we have $\id(a + a' \epsilon) = \id(a) + a' \epsilon$,
i.e. $\partial \id = 1$. For the constant functions we have $c(a + a' \epsilon) =
c(a) + 0 \epsilon$, i.e. $\partial c = 0$.
For addition, multiplication and composition of functions, we can derive the
following \emph{linearity}, \emph{product} and \emph{chain} rules:
\begin{align} \begin{split}
    (f + g)(a + a' \epsilon)
    \s &= \s (f + g)(a) \s + \s a' \times (\partial f + \partial g)(a) \epsilon
\end{split}\\ \begin{split}
    (f \times g)(a + a' \epsilon)
    \s &= \s (f \times g)(a) \s + \s a' \times (f \times \partial g \ + \ \partial f \times g)(a) \epsilon
\end{split}\\ \begin{split}
    (f \circ g)(a + a' \epsilon)
    \s &= \s (f \circ g)(a) \s + \s a' \times (\partial g \ \times \ \partial f \circ g)(a) \epsilon
\end{split} \end{align}

This generalises to smooth functions $\R^n \to \R^m$, where the partial
derivative $\partial_i$ is a derivation for each $i < n$.
The functions $\F_2^n \to \F_2^m$ on the two-element
field $\F_2$ with elementwise XOR as sum and conjunction as product
also forms a differential rig.
The partial derivative is given by $(\partial_i f)(\vec{x}) =
f(\vec{x}_{[x_i \mapsto 0]}) \oplus f(\vec{x}_{[x_i \mapsto 1]})$.
Intuitively, the $\F_2$ gradient $\partial_i f(\vec{x}) \in \F_2^m$ encodes
which coordinates of $f(\vec{x})$ actually depend on the input $x_i$.
An example of differential rig that isn't also a ring is given by the set
$\N[X]$ of polynomials with natural number coefficients, again each
partial derivative is a derivation.

A more exotic example is the rig of Boolean functions with elementwise
disjunction as sum and conjunction as product. Boolean functions $\B^n \to \B^m$
can be represented as tuples of $m$ propositional formulae over $n$ variables.
The partial derivative $\partial_i$ for $i < n$ is defined by induction over
the formulae: for variables we have $\partial_i x_j = \delta_{ij}$,
for constants $\partial_i 0 = \partial_i 1 = 0$ and for negation
$\partial_i \neg \phi = \neg \partial_i \phi$. The derivative of disjunctions
and conjunctions are given by the linerarity and product rules.
Equivalently, the gradient of a propositional formula can be given by
$\partial_i \phi = \neg \phi_{[x_i \mapsto 0]} \land \phi_{[x_i \mapsto 1]}$.
Concretely, a model satisfies $\partial_i \phi$ if and only if it satisfies
$\phi \leftrightarrow x_i$: the derivative is true when the variable and the
formula are positively correlated. Substituting $x_i$ with its negation,
we get that a model satisfies $\partial_i \phi_{[x_i \mapsto \neg x_i]}$
if and only if it satisfies $\phi \leftrightarrow \neg x_i$, i.e. iff variable
and formula are anti-correlated.
Note that although $\B$ and $\F_2$ are isomorphic as sets, they are distinct
rigs. Their derivations are related however by $\partial^{\F_2}_i f \mapsto
\partial^\B_i \phi \lor \partial^\B_i \phi_{[x_i \mapsto \neg x_i]}$
for $\phi : \B^n \to \B$ the formula corresponding to the function
$f : \F_2^n \to \F_2$. That is, a Boolean function depends on an input variable
precisely when either the corresponding formula is positively correlated or
anti-correlated.

Dual numbers are a fundamental tool for \emph{automatic differentiation}
\cite{Hoffmann16}, i.e.
they allow to compute the derivative of a function automatically from its
definition. The key idea is that given a definition of $f : \S^n \to \S^m$ as a
composition of elementary functions, we can compute $(\partial_i f)(a)$ by
evaluating $f(a + \epsilon)$ and projecting on the epsilon component.
