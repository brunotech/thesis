%!TEX root = ../../THESIS.tex

\section{Diagrammatic differentiation}

In this section, we introduce \emph{diagrammatic differentiation}: a graphical notation for computing the derivatives of parameterised diagrams.
On the theoretical side, we generalise the \emph{dual number construction} (discussed in section~\ref{1-dual-numbers}) from rigs to monoidal categories (section~\ref{2-dual-diagrams}).
We then apply this construction to the category of ZX diagrams (section~\ref{2b-differentiating-zx}) and of quantum circuits (section~\ref{3-dual-circuits}).
In section~\ref{4-bubbles} we give a formal definition of diagrams with bubbles and their gradient with the chain rule.
We use this to differentiate quantum circuits with neural networks as classical post-processing.
The theory comes with a DisCoPy implementation, gradients of classical-quantum circuits can then
be simplified using the PyZX library~\cite{KissingerVanDeWetering19}, compiled and executed
on quantum hardware with t$\vert$ket$\rangle$~\cite{SivarajahEtAl20}.

\subsection*{Related Work}

Penrose and Rindler~\cite{PenroseRindler84} used string diagrams to describe the geometry of space-time and an extra piece of notation is introduced: the covariant derivative is represented as a bubble around the tensor to be differentiated.
The same bubble notation for vector calculus has been proposed by Kim et al.~\cite{KimEtAl20}, but they have mainly pedagogical motivations and restrict themselves to the case of three-dimensional Euclidean space.
To the best of our knowledge, our definition is the first formal account of string diagrams with bubbles for derivatives.

Blute et al.~\cite{BluteEtAl06} axiomatised the notion of derivative with \emph{differential categories}.
More recently Cockett et al.~\cite{CockettEtAl19} generalised the notion of back-propagation with \emph{reverse derivative categories}, which have been proposed as a categorical foundation for gradient-based learning~\cite{CruttwellEtAl21}.
These frameworks all define the derivative of a morphism with respect to its domain.
In our setup however, we define the derivative of parametrised morphism with respect to parameters that are in some sense external to the category.
Investigating the relationship between these two definitions is left to future work.

The work presented in this section has its origin in Yeung's dissertation~\cite{Yeung20}, which focused on the diagrammatic differentiation of ansätze used in quantum machine learning.
In joint work with Yeung and de Felice~\cite{ToumiEtAl21a}, we gave this approach both its theoretical basis and its first implementation.
Wang and Yeung~\cite{WangYeung22} later generalised it from differentiation to the integration of ZX diagrams.
They also resolve a technical limitation of our approach: they show how to express any formal sum of ZX diagrams in terms of one diagram.
The same idea appeared independently in Jeandel et al.~\cite{JeandelEtAl22}, who also show how to apply our diagrammatic differentiation method to the Ising Hamiltonian, a key ingredient to many variational quantum algorithms~\cite{Hadfield21}.
