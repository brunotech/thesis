%!TEX root = ../THESIS.tex

\begin{abstract}
This thesis introduces a framework for quantum natural language processing (QNLP) based on a simple yet powerful analogy between computational linguistics and quantum mechanics: grammar as entanglement.
The grammar of a sentence connects the meaning of words in the same way that entanglement connects the states of quantum systems, they are both structures of information flow.
We turn this language-to-qubit analogy into an algorithm that maps the grammatical structure of sentences onto the architecture of parameterised quantum circuits.
We then use a hybrid classical-quantum algorithm to train the model on some data-driven task so that evaluating the circuits computes the meaning of sentences.
The implementation of these QNLP models led to the development of DisCoPy, a Python library for computing with string diagrams based on a premonoidal approach to computational category theory (Chapter 1).
We formalise our QNLP models as monoidal functors from grammar to quantum circuits and we introduce the idea of functorial learning, i.e. learning structure-preserving functors from diagram-like data (Chapter 2).
In order to learn optimal functor parameters via gradient descent, we introduce the notion of diagrammatic differentiation, a graphical calculus for computing the gradient of quantum circuits and string diagrams in general (Chapter 3).
\end{abstract}
