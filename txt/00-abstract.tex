%!TEX root = ../THESIS.tex

\begin{abstract}
This thesis presents a category-theoretic framework for natural language
processing on quantum computers, in short, QNLP. We map the grammatical
structure of sentences onto the architecture of parameterised quantum
circuits. We then optimise the parameters so that evaluating the
circuits computes the meaning of sentences in some data-driven task. We
formalise this language-to-qubit mapping in terms of parameterised
functors from a monoidal category of grammatical derivations to the
category of quantum circuits (Chapter 2). In order to learn the optimal
functor parameters via gradient descent, we introduce the notion of
diagrammatic differentiation as a generalisation of the dual number
construction from rings to monoidal categories (Chapter 3). The
implementation of these QNLP models led to the development of DisCoPy, a
Python library for computing with string diagrams based on a premonoidal
approach to computational category theory (Chapter 1).
\end{abstract}
