%!TEX root = main.tex

\section{Bubbles and the chain rule} \label{4-bubbles}

This section introduces an extension of the language of string diagrams
that encodes arbitrary non-linear operators on matrices: bubbles.
Previous sections already used two kinds of bubbles informally:
matrix exponentials and gradients. We give a formal definition of bubbles and
their gradients with the chain rule. We then use them to compute the gradient
of hybrid classical-quantum circuits where the measurement results can be
post-processed by any classical feed-forward neural network.

Fix a set of colours $C$.
Take a monoidal signature $\Sigma$, we construct the free monoidal category with
sums and bubbles $\mathbf{C}_{\beta(\Sigma)}^+$, i.e. arrows are formal sums of
diagrams with bubbles. We define the signature of bubbled diagrams
as a union $\beta(\Sigma) = \bigcup_{n \in \N} \beta(\Sigma, n)$ where the
signature of $(\leq n)$-nested bubbles $\beta(\Sigma, n)$ is defined by
induction:
$$
\beta(\Sigma, 0) \s = \s \Sigma \qquad \text{and} \qquad
\beta(\Sigma, n + 1) \s = \s \big\{\beta^c(d) : x \to y \s \vert \s
c \in C, \s d : x \to y \in \mathbf{C}_{\beta(\Sigma, n)}^+ \big\}$$
That is, we put a formal sum of diagrams $d \in \mathbf{C}_{\beta(\Sigma, n)}^+$
with $(\leq n)$-nested bubbles inside a $c$-coloured bubble and take it as a box
$\beta^c(d) \in \beta(\Sigma, n + 1)$ for diagrams with $(n + 1)$-nested
bubbles.
We say a monoidal category $\mathbf{C}$ has bubbles when it comes equipped with
a unary operator on homsets $\beta^c :
\coprod_{x, y} \mathbf{C}(x, y) \to \mathbf{C}(x, y)$ for each colour $c \in C$.

Although it makes the bureaucracy heavier, we may consider bubbles that change
the domain and codomain of the diagram inside. Such a bubble is defined by two
operators on objects
$\beta^c_\dom, \beta^c_\cod : \text{Ob}(\mathbf{C}) \to \text{Ob}(\mathbf{C})$
and an operator on homsets $\beta^c : \coprod_{x, y}
\mathbf{C}(x, y) \to \mathbf{C}(\beta^c_\dom(x), \beta^c_\cod(y))$.

\begin{example}
Bubbles first appear in Penrose and Rindler \cite{PenroseRindler84}
where they are used to encode the covariant derivative. An extra wire comes in
the bubble to encode the dimension of the tangent vector.
\end{example}

\begin{example}
The functorial boxes of Melli\`es \cite{Mellies06} can be thought of as
well-behaved bubbles, i.e. such that the composition of bubbles is the
bubble of the composition. Indeed, a functor $F : \mathbf{C} \to \mathbf{D}$
between two categories $\mathbf{C}$ and $\mathbf{D}$ defines a bubble on the
subcategory of their coproduct $\mathbf{C} \coprod \mathbf{D}$ spanned by $\mathbf{C}$.
\end{example}

\begin{example}
Bubbles appear under the name ``uooh'' (unary operator on homsets) in
\cite{HaydonSobocinski20} where they are used to encode the sep lines of
C.S.Peirce's existential graphs.
Take the predicates of a first-order logic as signature, i.e. one
generating object $x$ and each predicate $P$ with arity $k$ as a box with
$\dom(P) = 1$ and $\cod(P) = x^{\otimes k}$. Add generators for spiders to
encode lines of identity.
Then bubbled diagrams encode first-order logic formulae, and every formula can
be represented in this way. Logical deduction rules may be given entirely
in terms of diagrammatic rules.
The evaluation of first-order logic formulae is a bubble-preserving functor
$F : \mathbf{C}_{B(\Sigma)} \to \Mat_\B$, where bubbles are interpreted
as pointwise negation.
\end{example}

\begin{example}
Take colours to be arbitrary rig-valued functions $\S \to \S$, then
the category of matrices $\mathbf{Mat}_\S$ has bubbles given by pointwise
application. Gradient bubbles $\partial : \S \to \S$ are a special case.
\end{example}

\begin{example}
In the subcategory of square matrices, matrix exponential is an example of
bubble for $\S = \R, \C$. When $\S = \B$, square matrices are finite
graphs and reflexive transitive closure is an example.
\end{example}

\begin{example}
Bubbles can encode the standard non-linear operators used in machine learning.
The sigmoid $\sigma(x) = 1 / (1 + e^{-x})$ and rectified linear unit
$\sigma(x) = \max(0, x)$ are pointwise bubbles $\sigma : \R \to \R$.
The softmax function $\sigma : \R^n \to \R^n$ takes a vector $\vec{x}$,
applies exponential pointwise then normalises by $\sum_{i<n} e^{\vec{x}_i}$.
It can be drawn as a bubble around the diagram for the vector $\vec{x}$.
Bubbles may also depend on the labels from the dataset.
Take a loss function such as the relative entropy $l(\vec{y}, \vec{y}^\star)
= \sum_{i < m} \vec{y}_i \log(\vec{y}_i / \vec{y}^\star_i)$.
The partially-applied loss function $l(-, \vec{y}^\star) : \R^m \to \R$ for the
label $\vec{y}^\star \in \R^m$ can be drawn as a bubble around the diagram for
the prediction $\vec{y} \in \R^m$.
\end{example}

Bubbles compose by nesting, this defines a category of post-processes
$pp(\mathbf{C})$. The objects are pairs of objects from $\mathbf{C}$, arrows
$(x, y) \to (x', y')$ are $c$-coloured bubbles such that
$\beta^c_\dom(x) = x'$ and $\beta^c_\cod(y) = y'$.
If we apply this to the category of matrices, $pp(\mathbf{Mat}_\S)$ is the
category of all matrix-valued functions. In particular, this includes
any feed-forward neural networks. Indeed, take
$f = f_n \circ \dots \circ f_1 : \R^a \to \R^b$
where each layer is given by $f_i(\vec{x_i}) = \sigma(W_i \vec{x_i} + \beta_i)$
for the input vector $\vec{x_i} : 1 \to a_i$,
the parametrised weight matrix $W_i : a_i \to b_i$
and bias vector $\beta_i : 1 \to b_i$ in $\mathbf{Mat}_{\R^n \to \R}$
for $n$ the total number of parameters.
Drawing both the layer $f_i$ and the activation $\sigma : \R \to \R$ as bubbles
we get the following definition:
$$\tikzfig{4-1-neural-network}$$

When the bubble $\beta$ has a derivative $\partial \beta$, we may define the
gradient of bubbled diagrams with the chain rule
$\partial(\beta(f)) = (\partial \beta)(f) \times \partial f$.
In order to make sense of the multiplication, we assume that the homsets of our
category $\mathbf{C}$ have a product on homsets which is compatible with the
sum, i.e. each homset forms a rig\footnote{
We do not assume that products are compatible with composition,
in other words $\mathbf{C}$ need not be rig-enriched.}
and which commutes with the tensor,
i.e. $(f \times f') \otimes (g \times g')
= (f \otimes g) \times (f' \otimes g')$.
The category of matrices $\mathbf{Mat}_\S$ over a rig $\S$ is an example, each
homset $\mathbf{Mat}_\S(m, n)$ is a rig with entrywise sums and products.
Another example is the category of diagrams with spiders on each object, where
the product is given by pre/post-composition with the co/monoid structure.
We get the following equation:
$$\tikzfig{4-2-chain-rule}$$
For scalar diagrams, spiders are empty diagrams and the
equation simplifies to the usual chain rule.

Thus, we can draw both a parametrised quantum circuit and its classical
post-processing as one bubbled diagram in $cq(\mathbf{ZX}_n)$. By applying the
product rule to the quantum circuit and the chain rule to its post-processing,
we can compute a diagram for the overall gradient. This applies to
parametrised quantum circuits seen as machine learning models
\cite{BenedettiEtAl19}, to the patterns of measurement-based quantum
computing seen as ZX-diagrams \cite{DuncanPerdrix10} as well as the quantum
natural language processing of \cite{MeichanetzidisEtAl20a}.
