%!TEX root = main.tex

\section*{Conclusion, implementation \& future work}

We introduced diagrammatic differentiation for tensor calculus, using bubbles
to represent the partial derivative of a subdiagram. The product rule
allows to compute the gradient of a diagram from the gradient of its boxes.
Applying this to ZX diagrams, we showed how to compute the gradient of any linear
map with respect to a phase parameter. We then extended this to quantum circuits
with the parameter-shift rule and to neural networks with the chain rule.

Although this work focused on the theoretical foundations of diagrammatic
differentiation, we briefly describe its implementation as
part of the open-source DisCoPy library \cite{DeFeliceEtAl20}. A notebook
with examples is available in the documentation\footnote{
\href{https://discopy.readthedocs.io/en/main/notebooks/diag-diff.html}{
https://discopy.readthedocs.io/en/main/notebooks/diag-diff.html}}.
The \texttt{cqmap} module implements classical-quantum maps as NumPy arrays
\cite{VanDerWaltEtAl11}, with SymPy \cite{MeurerEtAl17} symbols as parameters.
The two modules \texttt{zx} and \texttt{circuit} build upon \texttt{monoidal},
the implementation of diagrams in monoidal categories. They both come with
an \texttt{eval} method which evaluates a diagram as a NumPy array and a
\texttt{grad} method which returns a formal sum of diagrams given a SymPy symbol.
The \texttt{zx} module comes with back-and-forth translations with the PyZX
library \cite{KissingerVanDeWetering19} for automated diagram simplification.
The \texttt{circuit} module interfaces with the tket compiler \cite{SivarajahEtAl20},
allowing to execute the diagrams for circuits and their gradient on quantum hardware.

For now, we have only defined gradients of diagrams with respect to one
parameter at a time. In future work, we plan to extend our definition to
compute the Jacobian of a tensor with respect to a vector of variables.
Other promising directions for research include the study of diagrammatic
differential equations, as well as a definition of integration for diagrams.

\section*{Acknowledgements}

The authors would like to thank the members of the Oxford quantum group for
their insightful feedback.
Special thanks go to Stefano Gogioso for developing the idea of
diagrammatic differentiation with RY during his MSc project \cite{Yeung20}.
We thank the QPL reviewers for their constructive feedback which improved the
presentation of this work.
We also thank Nicola Mariella for his contribution
to the implementation. AT thanks Simon Harrison for the Wolfson Harrison Quantum
Foundation Scholarship.
