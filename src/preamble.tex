%!TEX root = ../THESIS.tex

\usepackage[utf8]{inputenc}
\usepackage[english,greek]{babel}
% \usepackage{breakurl}

\usepackage{hyperref}
%WTF https://myjian.wordpress.com/2014/01/17/note-break-long-urls-with-xelatex/
\renewcommand{\UrlBreaks}{\do\/\do\-\do\_\do\.\do\a\do\b\do\c\do\d\do\e\do\f\do\g\do\h\do\i\do\j\do\k\do\l\do\m\do\n\do\o\do\p\do\q\do\r\do\s\do\t\do\u\do\v\do\w\do\x\do\y\do\z\do\A\do\B\do\C\do\D\do\E\do\F\do\G\do\H\do\I\do\J\do\K\do\L\do\M\do\N\do\O\do\P\do\Q\do\R\do\S\do\T\do\U\do\V\do\W\do\X\do\Y\do\Z}


\usepackage[
    style=alphabetic,
    sorting=nyt,
    backend=biber,
    maxbibnames=99,
    doi=true,
    isbn=false,
    url=true
]{biblatex}
\renewcommand*{\bibfont}{\raggedright\small}

% print url if no doi
\DeclareSourcemap{
  \maps[datatype=bibtex]{
    \map[overwrite]{
      \step[fieldsource=doi, final]
      \step[fieldset=url, null]
      \step[fieldset=urldate, null]
      \step[fieldset=eprint, null]
    }
  }
}
% print url if no arXiv
\DeclareSourcemap{
  \maps[datatype=bibtex]{
    \map[overwrite]{
      \step[fieldsource=eprint, final]
      \step[fieldset=url, null]
      \step[fieldset=urldate, null]
    }
  }
}

% https://tex.stackexchange.com/questions/163774/biblatex-print-bibliography-for-a-single-entry-within-an-enumeration
\DeclareBibliographyCategory{enumpapers}

\newcommand{\enumcite}[1]{%
  \nocite{#1}
  \addtocategory{enumpapers}{#1}%
  \defbibcheck{key#1}{
    \iffieldequalstr{entrykey}{#1}
      {}
      {\skipentry}}%
  \printbibliography[heading=none,check=key#1]%
}

\usepackage{amsthm, amssymb, amsmath}

\usepackage{fontspec}
\setmonofont{Menlo}[Scale=MatchLowercase]

\usepackage{setspace}
\renewcommand{\baselinestretch}{1.25}

\usepackage{minted}
\usemintedstyle{lovelace}
\setminted{frame=lines,fontseries=mono,fontsize=\footnotesize}
\setmintedinline{fontsize=auto}
\BeforeBeginEnvironment{minted}{\vspace{-16pt}}
\AfterEndEnvironment{minted}{\vspace{-8pt}}

\newcommand{\py}[1]{{\color{brown}\mintinline{python}{#1}}}

\newcommand{\inputpython}[1]{
{\vspace{-10pt} \footnotesize \inputminted[baselinestretch=1]{python}{#1}} \vspace{-10pt}}

\usepackage{caption}
\usepackage{subcaption}

\usepackage{graphicx}
\graphicspath{ {./img/} }

\usepackage{tikz}
\usepackage{tikz-cd}
\usepackage{src/tikzit}
\input{src/qcs.tikzstyles}

\usepackage[perpage]{footmisc}

\usepackage{ragged2e}
\usepackage{epigraph}
\setlength{\epigraphwidth}{0.5\textwidth}
\newcommand{\justepigraph}[2]{
\epigraph{\footnotesize \justifying #1}{#2}}


%!TEX root = ../THESIS.tex

\newcommand{\xto}[1]{\xrightarrow{#1}}
\newcommand{\sub}{\subseteq}
\newcommand{\s}{\enspace}
\newcommand{\eval}[1]{[ \! [ #1 ] \! ]}

\newcommand{\bra}[1]{\langle#1|}
\newcommand{\ket}[1]{|#1\rangle}
\newcommand{\braket}[2]{\langle#1|#2\rangle}

\newcommand{\dom}{\mathtt{dom}}
\newcommand{\cod}{\mathtt{cod}}
\newcommand{\id}{\mathtt{id}}
\newcommand{\then}{\mathtt{then}}
\newcommand{\tensor}{\mathtt{tensor}}
\newcommand{\len}{\mathtt{len}}

\renewcommand{\S}{\mathbb{S}}
\newcommand{\B}{\mathbb{B}}
\newcommand{\N}{\mathbb{N}}
\newcommand{\Z}{\mathbb{Z}}
\newcommand{\R}{\mathbb{R}}
\renewcommand{\C}{\mathbb{C}}

\def\fcmp{\mathbin{\raise 0.6ex\hbox{\oalign{\hfil$\scriptscriptstyle      \mathrm{o}$\hfil\cr\hfil$\scriptscriptstyle\mathrm{9}$\hfil}}}}


\newcommand{\downmapsto}{\rotatebox[origin=c]{-90}{$\mapsto$}\mkern2mu}


\newtheorem{definition}{Definition}[section]
\newtheorem{proposition}[definition]{Proposition}
\newtheorem{theorem}[definition]{Theorem}
\newtheorem{conjecture}[definition]{Conjecture}
\newtheorem{lemma}[definition]{Lemma}
\newtheorem{corollary}[definition]{Corollary}
\newtheorem{example}[definition]{Example}
\newtheorem{remark}[definition]{Remark}
\newtheorem{python}[definition]{Listing}
